%%%%%%%%%%%%%%%%%%%%%%%%%%%%%%%%%%%%%%%%%%
%%%%%%%%%%%%%%%%%%%%%%%%%%%%%%%%%%%%%%%%%%
%%%%%%%%%%%%%%%%%%%%%%%%%%%%%%%%%%%%%%%%%%
%%%%%%%%%%%%%%%%%%%%%%%%%%%%%%%%%%%%%%%%%%
\section{Classical Mereological Notions in the Sheaf-theoretic Setting}
\label{sec:classical-mereology-in-sheaves}

\noindent
In this section, we provide a discussion of what classical notions of mereology look like in the sheaf-theoretic setting.


%%%%%%%%%%%%%%%%%%%%%%%%%%%%%%%%%%%%%%%%%%
%%%%%%%%%%%%%%%%%%%%%%%%%%%%%%%%%%%%%%%%%%
\subsection{Standard Definitions}
\label{sec:standard-definitions}

\noindent
Recall the definitions of fusion and part.

\begin{Definition}[Fusions and parts]

We say that a section $s \in F(U)$ is a fusion iff there exists a cover $\{ U_{i} \}_{i \in I}$ of $U$ and a selection of patch candidates $\{ s_{i} \}_{i \in I}$ such that:

\[ 
\restrict{U}{U_{i}}(s) = s_{i}, \quad \text{ for each } U_{i}.
\]

\noindent
Given $t \in F(V)$ and $s \in F(U)$ with $V \childOf/ U$ and $V \not = \bottom/$, we say $t$ is a part of $s$, denoted $t \partOf/ s$, iff $s$ is a fusion and

\[
\restrict{U}{V}(s) = t.
\]

\end{Definition}

\noindent
Morally, the overlap of two fusions is a shared part. In this context, it is a section used in two gluings.

\begin{Definition}[Overlap]

Let $s \in F(V)$ and $t \in F(U)$ be fusions. We say that $s$ and $t$ overlap, denoted $s \partOverlap/ t$, iff

\begin{itemize}

\item there exists a region $W$ such that $W \childOf/ (V \meet/ U)$ and $W \not = \bottom/$,
\item a section $r \in F(W)$

\end{itemize}

\noindent
such that $r \partOf/ F(V)$ and $r \partOf/ F(U)$.

\end{Definition}

\noindent
Note that overlap is not merely order-theoretic: it is not ``just'' a shared region. In this setting, two fusions can have a shared region without having a shared part in that region. Parts are only those sections that comprise a gluing.

Disjointness in fusions amounts to disjointness in the regions they occupy: 

\begin{Theorem}[Regional disjointness]

Let $s \in F(V)$ and $t \in F(U)$ be fusions. Then:

\[
  \lnot (s \partOverlap/ t) \implies V \meet/ U = \bottom/.
\]

\end{Theorem}

\begin{proof}

\todo{Admitted.}
\end{proof}

\noindent
A proper part is a part that is not identical to its fusion.

\begin{Definition}[Proper part]

Let $s \in F(V)$ and $t \in F(U)$. We say that $s$ is a proper part of $t$, denoted $s \properPartOf/ t$, iff

\begin{itemize}

\item $s \partOf/ t$,
\item $V \not = U$.

\end{itemize}

\end{Definition}

\begin{Theorem}[Nothing is a proper part of itself]

For any part $s$, $s \not \properPartOf/ s$.

\end{Theorem}

\begin{proof}

By reflexivity, $s \partOf/ s$. But $s \in F(U)$ for some $U$, and $U = U$, so $s \not \properPartOf/ s$.
\end{proof}



%%%%%%%%%%%%%%%%%%%%%%%%%%%%%%%%%%%%%%%%%%
%%%%%%%%%%%%%%%%%%%%%%%%%%%%%%%%%%%%%%%%%%
\subsection{Partial Ordering}
\label{sec:partial-order}

\noindent
In the sheaf-theoretic setting, parthood is a partial order.

\begin{Theorem}[Reflexivity]

For any part $s$, $s \partOf/ s$.

\end{Theorem}

\begin{proof}

For any $U$, $\{ U \}$ is its trivial cover. For any $s \in F(U)$, $\{ s \}$ is a trivial selection of patch candidates for that trivial cover $\{ U \}$. Further, $\restrict{U}{U}(s) = s$, since restricting to the same region is an identity. Hence, $s$ is a fusion of itself, and $s \partOf/ s$, as required.
\end{proof}

\begin{Theorem}[Transitivity]

For any parts $s$, $t$, $u$, if $s \partOf/ t$ and $t \partOf/ u$, then $s \partOf/ u$.

\end{Theorem}

\begin{proof}

Suppose $s \partOf/ t$ and $t \partOf/ u$, with $s \in F(W)$, $t \in F(V)$, and $u \in F(U)$. Then $\restrict{U}{V}(u) = t$, and $\restrict{V}{W}(t) = s$. By transitivity of restriction, $\restrict{U}{W}(u) = s$, and hence $s \partOf/ u$. 
\end{proof}

\noindent
\begin{Theorem}[Antisymmetry]

For any parts $s$, $t$, if $s \partOf/ t$ and $t \partOf/ s$, then $s = t$.

\end{Theorem}

\begin{proof}

Suppose $s \partOf/ t$ and $t \partOf/ s$, with $s \in F(U)$ and  $t \in F(V)$.  Since $s \partOf/ t$ and $t \in F(V)$, there is a region $W$ such that $s \in F(W)$, $W \childOf/ V$, and $\restrict{V}{W}(t) = s$. But since we already have that $s \in F(U)$, it must be that $U = W$. Substituting $U$ for $W$ in $W \childOf/ V$ and $\restrict{V}{W}(t) = s$ yields $U \childOf/ V$ and $\restrict{V}{U}(t) = s$.

Conversely, since $t \partOf/ s$, by a similar argument, there is a region $Z$ such that $V = Z$, and substituting $V$ for $Z$ yields $V \childOf/ U$ and $\restrict{U}{V}(s) = t$. 

Since $U \childOf/ V$ and $V \childOf/ U$, it must be that $U = V$. If we then substitute $U$ for $V$ in  $\restrict{V}{U}(t) = s$ and $\restrict{U}{V}(s) = t$, we get $\restrict{U}{U}(t) = s$ and $\restrict{U}{U}(s) = t$. But $\restrict{U}{U}$ is the identity, so $t = s$, as required.
\end{proof}


%%%%%%%%%%%%%%%%%%%%%%%%%%%%%%%%%%%%%%%%%%
%%%%%%%%%%%%%%%%%%%%%%%%%%%%%%%%%%%%%%%%%%
\subsection{Extensionality}
\label{sec:extensionality}

\noindent
In the sheaf-theoretic setting, extensionality says that fusions are identical when they are glued from the same patch candidates. Formally:

\begin{Definition}[Extensionality]

We say that extensionality holds in a presheaf $F$ iff, for all fusions $s$, $t$ in $F$:

\[
  (\forall r, r \partOf/ t \Longleftrightarrow r \partOf/ s) \implies s = t.
\]

\end{Definition}

\noindent
If extensionality holds, then equal gluings must live in the same fiber.

\begin{Theorem}[Equality in fibers]

If $s \in F(V)$ and $t \in F(U)$ are gluings and $s = t$, then $U = V$.

\end{Theorem}

\begin{proof}

Suppose $s \in F(V)$, $t \in F(U)$, and $s = t$. Since $s$ is a fusion, there exists a cover $\{ V_{i} \}$ and selection of patch candidates $\{ s_{i} \}_{i \in I}$ such that $\restrict{V}{V_{i}} = s_{i}$ for every $i \in I$.  But since $s = t$, if we substitute $t$ for $s$, we get $\restrict{U}{V_{i}}(t) = s_{i}$, for every $i \in I$. 

Since $t$ restricts to each region $U_{i}$ in the cover, it follows that $U_{i} \childOf/ V$, for all $i \in I$. But since $\{ U_{i} \}_{i \in I}$ is a cover of $U$, $U$ is their join:

\[
  U = \bigjoin/_{i \in I} U_{i}.
\]

\noindent
Since every $U_{i}$ is below $V$, it follows that the join of the cover's components is also below $V$, for the join of any collection of regions is their least upper bound, hence, $V$ is guaranteed to be no lower than that join. Hence $U \childOf/ V$.

Going the other way, by a similar argument, we can show that $V \childOf/ U$. Then, by antisymmetry, $V = U$.
\end{proof}

\noindent
Extensionality can fail in presheaves.

\begin{Theorem}[Extensionality failure in presheaves]

It is not the case that extensionality holds in every presheaf.

\end{Theorem}

\begin{proof}

In the presheaf from \cref{ex:non-monopresheaf}, $b$ and $c$ are glued from the same parts, yet $b \not = c$.
\end{proof}

\noindent
By contrast, monopresheaves and sheaves have extensional gluings.

\begin{Theorem}[Extensionality in monopresheaves and sheaves]

Let $F$ be a presheaf over a locale. If $F$ is a monosheaf or a sheaf, then extensionality holds in $F$.

\end{Theorem}

\begin{proof}

Let $F$ be a monoprsheaf, and let $s, t \in F(U)$ be fusions such that

\[
  \forall r, r \partOf/ s \Longleftrightarrow r \partOf/ t.
\]

\noindent
Then for any cover $\{ U_{i} \}_{i \in I}$ of $U$, every patch $r_{i}$ used to glue $s$ is also used to glue $t$. Thus, $\restrict{U}{U_{i}}(s) = \restrict{U}{U_{i}}(t)$, for each $U_{i}$ in the cover. 

By the definition of monopresheaves, it follows that $s = t$. Since a sheaf is a monopresheaf with extra conditions, the same argument shows that extensionality holds for sheaves.
\end{proof}


%%%%%%%%%%%%%%%%%%%%%%%%%%%%%%%%%%%%%%%%%%
%%%%%%%%%%%%%%%%%%%%%%%%%%%%%%%%%%%%%%%%%%
\subsection{Supplementation}
\label{sec:supplementation}

\noindent
Supplementation is the idea that fusions are not made from a single proper part. If you remove a proper part from a fusion, there should be at least one other proper part left over.

There are weaker and stronger formulations. In this setting, weak supplementation is the claim that if $s$ is a proper part of $t$, then $t$ has another part $r$ disjoint from $s$. 

\begin{Definition}[Weak supplementation]

We say that weak supplementation holds in a presheaf $F$ iff, for any part $s \in F(V)$ and fusion $t \in F(U)$:

\[
  s \properPartOf/ t \implies \exists r \in F(W) (r \partOf/ t \text{ and } \lnot(r \partOverlap/ s)).
\]

\end{Definition}

\noindent
Note that disjointness of fusions $r$ and $s$ implies disjointness of regions $W$ and $V$, while $s \properPartOf/ t$ means that $V \childOfStrict/ U$. Thus, for weak supplementation to hold, there must be another region $W$ disjoint from $V$. This is purely a requirement on the available regions.

\begin{Definition}[Regional supplementation]

We say that a locale $\category{L}$ is regionally supplemented iff, for all $U \in \category{L}$ and all $V \childOfStrict/ U$ with $V \not = \bottom/$, there exists a $W \not = \bottom/$ such that

\[
  W \childOf/ U
  \quad \text{ and } \quad
  W \meet/ V = \bottom/.
\]

\end{Definition}

\noindent
In other words, in order for weak supplementation to hold in a presheaf, the underlying locale must be regionally supplemented.

Regional supplementation holds in Boolean locales. 

\todo{Cite Johnstone or Goldblatt for Boolean locales, e.g. complement pushing to top and bottom via join and meet.}

\begin{Theorem}[Boolean locales are regionally supplemented]

Let $\category{L}$ be a locale. If $\category{L}$ is Boolean, then $\category{L}$ is regionally supplemented.

\end{Theorem}

\begin{proof}

Suppose $\category{L}$ is Boolean, and fix a $V \childOfStrict/ U$ with $V \not = \bottom/$. Since $\category{L}$ is Boolean, $V$ has a complement $\lnot V$ satisfying

\[
  V \meet/ \lnot V = \bottom/
  \quad \text{ and } \quad
  V \join/ \lnot V = \top.
\]

\noindent
Define $W = U \meet/ \lnot V$. Then:

\begin{enumerate}
  \item 
    In a locale, $a \meet/ b \childOf/ a$. Let $a = U$ and $b = \lnot V$. Then
    $U \meet/ \lnot V \childOf/ U$. Substituting $W$ yields $W \childOf/ U$.
  \item 
    $W \meet/ V$ = $(U \meet/ \lnot V) \meet/ V$ = 
    $U \meet/ (\lnot V \meet/ V)$ = $U \meet/ \bottom/ = \bottom/$.
  \item
    In a Boolean locale, $a \meet/ b = \bottom/ \Longleftrightarrow a \childOf/ \lnot b$. 
    If we assume for contradiction that $U \meet/ \lnot V = \bottom/$, it therefore follows that
    $U \childOf/ \lnot(\lnot V)$. But since $\lnot \lnot a = a$ in a Boolean locale, 
    $U \childOf/ V$. That contradicts the assumption $V \childOfStrict/ U$.
    Hence, $W = U \meet/ \lnot V \lnot = \bottom/$.
\end{enumerate}

\noindent
Thus, $W$ witnesses regional supplementation.
\end{proof}

\noindent
So, Boolean locales are regionally supplemented. It goes the other direction too.

\begin{Theorem}[Regionally supplemented locales are Boolean]

Let $\category{L}$ be a locale. If $\category{L}$ is regionally supplemented, then $\category{L}$ is Boolean.

\end{Theorem}

\begin{proof}

To prove that $\category{L}$ is Boolean, it suffices to show that every $V \in \category{L}$ has a complement.

We construct a candidate complement. Suppose $\category{L}$ is regionally supplemented, and fix a $V \in \category{L}$. Next, define:

\[
  \mathcal D_{V} = \{ X \in \category{L} \mid X \meet/ V = \bottom/ \},
\]

\noindent
i.e., the set of all regions disjoint from $V$. This set is nonempty, since it at least contains $\bottom/$. Then, define the complement of $V$ as the largest element of $D_{V}$, which is obtained by taking their join:

\[
  \lnot V = \bigjoin/ \mathcal D_{V}.
\]

\noindent
This join exists because $\category{L}$ is a complete lattice.

Having constructed a candidate complement of $V$, we next show that is the complement by showing that it satisfies the Boolean complement laws, namely that $V \meet/ \lnot V = \bottom/$ and $V \join/ \lnot V = \top$.

First, we show that $V \meet/ \lnot V = \bottom/$. We can do this by the distributivity of locales:

\[
  V \meet/ \lnot V = V \meet/ \bigjoin/_{X \in \mathcal D_{V}} X = \bigjoin/_{X \in \mathcal D_{V}} (V \meet/ X) = \bigjoin/ \bottom/ = \bottom/.
\]

\noindent
In other words, $\lnot V$ is disjoint from $V$.

Next, we must show that $V \join/ \lnot V = \top$. Assume, for contradiction, that $V \join/ \lnot V \childOfStrict/ \top$. Let $U$ be $V \join/ \lnot V$. Then $V \childOfStrict/ U \childOfStrict/ \top$.

Regional supplementation says that, since $V \childOfStrict/ U$, there exists a $W \not = \bottom/$ such that $W \childOf/ U$ and $W \meet/ V = \bottom/$. 

But $W \meet/ V = \bottom/$ implies that $W \in \mathcal D_{V}$. So, $W \childOf/ \bigjoin/ \mathcal D_{V} = \lnot V$. In a locale, if $a \childOf/ b$ and $a \childOf/ c$, then $a \childOf/ b \meet/ c$. Here, we have that $W \childOf/ U$ and $W \childOf/ \lnot V$. Thus, $W \childOf/ U \meet/ \lnot V$. But then:

\[
  W \childOf/ U \meet/ \lnot V = (V \join/ \lnot V) \meet/ \lnot V = \lnot V.
\]

\noindent
That implies that $W \childOf/ \lnot V \childOf/ U$, so $W$ was already below $\lnot V$.

Now observe that $\lnot V \childOf/ V \join/ \lnot V = U$, so adding $W$ below $\lnot V$ cannot enlarge $U$. But regional supplementation requires that $W \not =\bottom/$ and $W \childOf/ U$ strictly witnessing supplementation inside $U$. This contradicts the assumption that $\lnot V$ already collected \emph{all} elements disjoint from $V$. 

Formally, if $V \join/ \lnot V \childOfStrict/ \top$, regional supplementation produces a non-bottom disjoint region below $U$, but by construction all such regions are already $\childOf/ \lnot V$, forcing $U = V \join/ \lnot \top$, which is a contradiction. Hence, $V \join/ \lnot V = \top$.

Since $V \meet/ \lnot V = \bottom/$ and $V \join/ \lnot V = \top$, $\lnot V$ is a Boolean complement. Since $V$ was chosen arbitrarily, this holds for all $V$ in $\category{L}$, hence $\category{L}$ is a Boolean locale.
\end{proof}

\noindent
Thus, regional supplementation is equivalent to the Booleanness of the locale, and we can conclude that a presheaf can satisfy weak supplementation only if it has the right kind of geometry, namely a Boolean geometry.

\begin{Theorem}[Booleanness = regional supplementation]

Let $\category{L}$ be a locale. Then the following are equivalent:

\begin{enumerate}
  \item $\category{L}$ is regionally supplemented.
  \item $\category{L}$ is Boolean.
  \item For any presheaf $F$ over $\category{L}$, $F$ can satisfy weak supplementation only if $\category{L}$ is regionally supplemented or Boolean.
\end{enumerate}

\end{Theorem}

\todo{Tighten up 3. What's that really mean? Maybe keep it outside the theorem, and keep it meta.}

\begin{proof}
Immediate.
\end{proof}

\noindent
This strongly suggests that classical supplementation is not merely a mereological axiom. It presupposes a Boolean geometry of parts. This explains why supplementation can fail in topological contexts (e.g. in $Sh([0, 1])$), since topological contexts are typically non-Boolean, and it explains why supplementation tacitly assumes a Boolean background.

 