%%%%%%%%%%%%%%%%%%%%%%%%%%%%%%%%%%%%%%%%%%
%%%%%%%%%%%%%%%%%%%%%%%%%%%%%%%%%%%%%%%%%%
%%%%%%%%%%%%%%%%%%%%%%%%%%%%%%%%%%%%%%%%%%
%%%%%%%%%%%%%%%%%%%%%%%%%%%%%%%%%%%%%%%%%%
\section{Classical Mereological Notions in the Sheaf-theoretic Setting}
\label{sec:classical-mereology}

\noindent
In this section, we show how classical mereological notions translate to the sheaf-theoretic setting.


%%%%%%%%%%%%%%%%%%%%%%%%%%%%%%%%%%%%%%%%%%
%%%%%%%%%%%%%%%%%%%%%%%%%%%%%%%%%%%%%%%%%%
\subsection{Standard Definitions}
\label{sec:standard-definitions}

\noindent
Recall the definitions of fusion and part.

\begin{Definition}[Fusions and parts]

We say that a section $s \in F(U)$ is a fusion iff there exists a cover $\{ U_{i} \}_{i \in I}$ of $U$ and a selection of patch candidates $\{ s_{i} \}_{i \in I}$ such that:

\[ 
\restrict{U}{U_{i}}(s) = s_{i}, \quad \text{ for each } U_{i}.
\]

\noindent
Given $t \in F(V)$ and $s \in F(U)$ with $V \childOf/ U$ and $V \not = \bottom/$, we say $t$ is a part of $s$, denoted $t \partOf/ s$, iff $s$ is a fusion and

\[
\restrict{U}{V}(s) = t.
\]

\end{Definition}

\noindent
Morally, the overlap of two fusions is a shared part. In this context, it is a section used in two gluings.

\begin{Definition}[Overlap]

Let $s \in F(V)$ and $t \in F(U)$ be fusions. We say that $s$ and $t$ overlap, denoted $s \partOverlap/ t$, iff

\begin{itemize}

\item there exists a region $W$ such that $W \childOf/ (V \meet/ U)$ and $W \not = \bottom/$,
\item a section $r \in F(W)$

\end{itemize}

\noindent
such that $r \partOf/ F(V)$ and $r \partOf/ F(U)$.

\end{Definition}

\noindent
Note that overlap is not merely order-theoretic: it is not ``just'' a shared region. In this setting, two fusions can have a shared region without having a shared part in that region. Parts are only those sections that comprise a gluing.

If fusions overlap, the regions they occupy overlap.

\begin{Theorem}[Regional overlap]

Let $s \in F(V)$ and $t \in F(U)$ be fusions. Then:

\[
  s \partOverlap/ t \implies V \meet/ U \not = \bottom/.
\]

\end{Theorem}

\begin{proof}

Suppose $s \partOverlap/ t$. By definition, there exists a region $W$ such that $\bottom/ \strictChildOf/ W \childOf/ (V \meet/ U)$ with a section $r \in F(W)$ that is a part of both $s$ and $t$. Since $W$ is strictly larger than $\bottom/$ and $W \childOf/ V \meet/ U$, then $V \meet/ U \not = \bottom/$.
\end{proof}

\noindent
The converse is not true. Fusions can fail to overlap either because they occupy disjoint regions, or because they don't have a part in a shared region. So geometric overlap is sufficient but not necessary for mereological disjointness.

A proper part is a part that is not identical to its fusion.

\begin{Definition}[Proper part]

Let $s \in F(V)$ and $t \in F(U)$. We say that $s$ is a proper part of $t$, denoted $s \properPartOf/ t$, iff

\begin{itemize}

\item $s \partOf/ t$,
\item $V \not = U$.

\end{itemize}

\end{Definition}

\begin{Theorem}[Nothing is a proper part of itself]

For any part $s$, $s \not \properPartOf/ s$.

\end{Theorem}

\begin{proof}

By reflexivity, $s \partOf/ s$. But $s \in F(U)$ for some $U$, and $U = U$, so $s \not \properPartOf/ s$.
\end{proof}



%%%%%%%%%%%%%%%%%%%%%%%%%%%%%%%%%%%%%%%%%%
%%%%%%%%%%%%%%%%%%%%%%%%%%%%%%%%%%%%%%%%%%
\subsection{Partial Ordering}
\label{sec:partial-order}

\noindent
In the sheaf-theoretic setting, parthood is a partial order.

\begin{Theorem}[Reflexivity]

For any part $s$, $s \partOf/ s$.

\end{Theorem}

\begin{proof}

For any $U$, $\{ U \}$ is its trivial cover. For any $s \in F(U)$, $\{ s \}$ is a trivial selection of patch candidates for that trivial cover $\{ U \}$. Further, $\restrict{U}{U}(s) = s$, since restricting to the same region is an identity. Hence, $s$ is a fusion of itself, and $s \partOf/ s$, as required.
\end{proof}

\begin{Theorem}[Transitivity]

For any parts $s$, $t$, $u$, if $s \partOf/ t$ and $t \partOf/ u$, then $s \partOf/ u$.

\end{Theorem}

\begin{proof}

Suppose $s \partOf/ t$ and $t \partOf/ u$, with $s \in F(W)$, $t \in F(V)$, and $u \in F(U)$. Then $\restrict{U}{V}(u) = t$, and $\restrict{V}{W}(t) = s$. By transitivity of restriction, $\restrict{U}{W}(u) = s$, and hence $s \partOf/ u$. 
\end{proof}

\noindent
\begin{Theorem}[Antisymmetry]

For any parts $s$, $t$, if $s \partOf/ t$ and $t \partOf/ s$, then $s = t$.

\end{Theorem}

\begin{proof}

Suppose $s \partOf/ t$ and $t \partOf/ s$, with $s \in F(U)$ and  $t \in F(V)$.  Since $s \partOf/ t$ and $t \in F(V)$, there is a region $W$ such that $s \in F(W)$, $W \childOf/ V$, and $\restrict{V}{W}(t) = s$. But since we already have that $s \in F(U)$, it must be that $U = W$. Substituting $U$ for $W$ in $W \childOf/ V$ and $\restrict{V}{W}(t) = s$ yields $U \childOf/ V$ and $\restrict{V}{U}(t) = s$.

Conversely, since $t \partOf/ s$, by a similar argument, there is a region $Z$ such that $V = Z$, and substituting $V$ for $Z$ yields $V \childOf/ U$ and $\restrict{U}{V}(s) = t$. 

Since $U \childOf/ V$ and $V \childOf/ U$, it must be that $U = V$. If we then substitute $U$ for $V$ in  $\restrict{V}{U}(t) = s$ and $\restrict{U}{V}(s) = t$, we get $\restrict{U}{U}(t) = s$ and $\restrict{U}{U}(s) = t$. But $\restrict{U}{U}$ is the identity, so $t = s$, as required.
\end{proof}


%%%%%%%%%%%%%%%%%%%%%%%%%%%%%%%%%%%%%%%%%%
%%%%%%%%%%%%%%%%%%%%%%%%%%%%%%%%%%%%%%%%%%
\subsection{Extensionality}
\label{sec:extensionality}

\noindent
In the sheaf-theoretic setting, extensionality says that fusions are identical when they are glued from the same patch candidates. Formally:

\begin{Definition}[Extensionality]

We say that extensionality holds in a presheaf $F$ iff, for all fusions $s$, $t$ in $F$:

\[
  (\forall r, r \partOf/ t \Longleftrightarrow r \partOf/ s) \implies s = t.
\]

\end{Definition}

\noindent
If extensionality holds, then equal gluings must live in the same fiber.

\begin{Theorem}[Equality in fibers]

If $s \in F(V)$ and $t \in F(U)$ are gluings and $s = t$, then $U = V$.

\end{Theorem}

\begin{proof}

Suppose $s \in F(V)$, $t \in F(U)$, and $s = t$. Since $s$ is a fusion, there exists a cover $\{ V_{i} \}$ and selection of patch candidates $\{ s_{i} \}_{i \in I}$ such that $\restrict{V}{V_{i}} = s_{i}$ for every $i \in I$.  But since $s = t$, if we substitute $t$ for $s$, we get $\restrict{U}{V_{i}}(t) = s_{i}$, for every $i \in I$. 

Since $t$ restricts to each region $U_{i}$ in the cover, it follows that $U_{i} \childOf/ V$, for all $i \in I$. But since $\{ U_{i} \}_{i \in I}$ is a cover of $U$, $U$ is their join:

\[
  U = \bigjoin/_{i \in I} U_{i}.
\]

\noindent
Since every $U_{i}$ is below $V$, it follows that the join of the cover's components is also below $V$, for the join of any collection of regions is their least upper bound, hence, $V$ is guaranteed to be no lower than that join. Hence $U \childOf/ V$.

Going the other way, by a similar argument, we can show that $V \childOf/ U$. Then, by antisymmetry, $V = U$.
\end{proof}

\noindent
Extensionality can fail in presheaves.

\begin{Theorem}[Extensionality in presheaves]

It is not the case that extensionality holds in every presheaf.

\end{Theorem}

\begin{proof}

In the presheaf from \cref{ex:non-monopresheaf}, $b$ and $c$ are glued from the same parts, yet $b \not = c$.
\end{proof}

\noindent
By contrast, monopresheaves and sheaves have extensional gluings.

\begin{Theorem}[Extensionality in monopresheaves and sheaves]

Let $F$ be a presheaf over a locale. If $F$ is a monosheaf or a sheaf, then extensionality holds in $F$.

\end{Theorem}

\begin{proof}

Let $F$ be a monoprsheaf, and let $s, t \in F(U)$ be fusions such that

\[
  \forall r, r \partOf/ s \Longleftrightarrow r \partOf/ t.
\]

\noindent
Then for any cover $\{ U_{i} \}_{i \in I}$ of $U$, every patch $r_{i}$ used to glue $s$ is also used to glue $t$. Thus, $\restrict{U}{U_{i}}(s) = \restrict{U}{U_{i}}(t)$, for each $U_{i}$ in the cover. 

By the definition of monopresheaves, it follows that $s = t$. Since a sheaf is a monopresheaf with extra conditions, the same argument shows that extensionality holds for sheaves.
\end{proof}


%%%%%%%%%%%%%%%%%%%%%%%%%%%%%%%%%%%%%%%%%%
%%%%%%%%%%%%%%%%%%%%%%%%%%%%%%%%%%%%%%%%%%
\subsection{Supplementation}
\label{sec:supplementation}

\noindent
In the sheaf-theoretic setting, weak supplementation says that fusions are glued from more than one patch candidate. 

\begin{Definition}[Weak supplementation]

We say that weak supplementation holds in a presheaf $F$ iff, for any $s \in F(V)$ and $t \in F(U)$:

\[
  s \properPartOf/ t \implies \exists W \in \category{L}, r \in F(W) (r \properPartOf/ t \text{ and } \lnot(r \partOverlap/ s)).
\]

\end{Definition}

\noindent
In the sheaf-theoretic setting, weak supplementation need not hold. Consider a simple locale $\category{L}$ with $\bottom/ \strictChildOf/ V \strictChildOf/ U$. Then define a presheaf $F$ such that $F(\bottom/) = \{ 0 \}$, $F(V) = \{ s \}$, and $F(U) = \{ t \}$. This is (trivially) a presheaf, monopresheaf, and a sheaf, yet weak supplementation fails.

Strong supplementation says that if a section $s$ lives in a fiber outside of a fusion $t$, it is disjoint from one of the patch candidates from which $t$ is glued.

\begin{Definition}[Strong supplementation]

We say that strong supplementation holds in a presheaf $F$ iff, for any fusions $s \in F(V)$ and $t \in F(U)$:

\[
  s \not \partOf/ t \implies \exists W \in \category{L}, r \in F(W) (r \properPartOf/ t \text{ and } \lnot(r \partOverlap/ s)).
\]

\end{Definition}

\noindent
This is a strictly stronger condition than weak supplementation, so it fails in the sheaf-theoretic setting too.

It may be tempting to suppose that mereological supplementation imposes a corresponding supplementation condition on the underlying locale. However, this inference is invalid in the sheaf-theoretic setting. Two fusions may fail to overlap either because their regions are disjoint or because, despite regional overlap, no part of either fusion occupies the overlapping region.


%%%%%%%%%%%%%%%%%%%%%%%%%%%%%%%%%%%%%%%%%%
%%%%%%%%%%%%%%%%%%%%%%%%%%%%%%%%%%%%%%%%%%
\subsection{Unrestricted fusion}
\label{sec:unrestricted-fusion}
 
 \noindent
There is a sense in which unrestricted fusion does not make sense in the sheaf-theoretic setting. Consider the following.

\begin{itemize}

\item In the classical-setting, everything in the domain is already a part. By contrast, in the sheaf-theoretic setting, not every section of a fiber is a part. Only those sections that glue are parts. 

\item In the classical setting, you can collect together any plurality of parts as candidates for a fusion. By contrast, in the sheaf-theoretic setting, you can't pick just any selection of patch candidates. You must select them from the regions in a cover.

\item In the classical setting, once you select a plurality of parts, no further coherence or compatibility conditions need to be met before fusing them. By contrast, in the sheaf-theoretic setting, a selection of patch candidates can be glued only if they are compatible.

\end{itemize}

If one really wanted to translate unrestricted fusion into the sheaf-theoretic setting, it would have to be stated along the following lines:

\begin{Definition}[Sheaf-theoretic unrestricted fusion]

Given a presheaf $F$ over a local $\category{L}$, for any region $U \in \category{L}$ and selection of sections $\{ s_F(U_{i}) \}_{i \in I}$ satisfying

\begin{itemize}

\item $\{ U_{i} \}_{i \in I}$ covers $U$
\item $\{ s_F(U_{i}) \}_{i \in I}$ are compatible

\end{itemize}

\noindent
There exists a section $s \in F(U)$ such that $\restrict{U}{U_{i}}(s) = s_{i}$ for each $i$. 

\end{Definition}

\noindent
But that is just the sheaf condition. In the sheaf-theoretic setting, unrestricted fusion thus amounts to requiring that all part-whole complexes are sheaves. That is a strong requirement. As we have seen, many natural part-whole complexes are more naturally modeled with the looser gluing conditions of monopresheaves and presheaves.