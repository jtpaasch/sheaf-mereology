\begin{flushright}
  \small{The ships hung in the sky in much the same way that bricks don’t. \\
  -- \emph{The Hitchhiker's Guide to the Galaxy}}
\end{flushright}

%%%%%%%%%%%%%%%%%%%%%%%%%%%%%%%%%%%%%%%%%%
%%%%%%%%%%%%%%%%%%%%%%%%%%%%%%%%%%%%%%%%%%
%%%%%%%%%%%%%%%%%%%%%%%%%%%%%%%%%%%%%%%%%%
%%%%%%%%%%%%%%%%%%%%%%%%%%%%%%%%%%%%%%%%%%
%%%%%%%%%%%%%%%%%%%%%%%%%%%%%%%%%%%%%%%%%%
%%%%%%%%%%%%%%%%%%%%%%%%%%%%%%%%%%%%%%%%%%
%%%%%%%%%%%%%%%%%%%%%%%%%%%%%%%%%%%%%%%%%%
%%%%%%%%%%%%%%%%%%%%%%%%%%%%%%%%%%%%%%%%%%
\section{Introduction}
\label{sec:introduction}

\noindent
Category theory is firmly embedded in contemporary mathematics, but it remains much less familiar within contemporary philosophy (\cite{Landry:2018}, \cite{Sica:2006}). Topos theory---a branch of category theory---is less familiar still (\cite{Badiou:2014}, \cite{Badiou:2009}, \cite{Bhattacharyya:2012}). Nevertheless, since the 1960s, topos theorists have uncovered deep connections between geometry, logic, and structure.

In this paper, I convey one of the central insights of topos theory: first-order predicate logic is not about truth in the abstract. It is about how parts fit together.\footnote{Stated technically: the subobject functor is representable. See \cite[pp.~57--58]{Bell:1988}, \cite[p.~139]{LambekAndScott:1986}, \cite[pp.~289--290]{Borceux:1994:vol3}, \cite[pp.~33--34]{MacLaneAndMoerdijk:1994}.}

This contrasts with a common picture of logic as a fixed system of truth values independent of the kinds of objects to which it is applied. Topos theory challenges this, showing instead that the logic appropriate to a given mathematical universe arises from the ways in which objects in that universe can have parts, overlap, and combine.

The aim of this expository paper is to make the idea concrete. No prior knowledge of category theory is assumed. Definitions are illustrated with worked examples, so that anyone familiar with first-order logic and basic set-theoretic notation can follow along with pencil and paper. References to standard texts are given for the mathematically inclined.

Relating topos theory to mereology in the literature is rare. Notably, Thomas Mormann's ``structural mereology'' (\cite{Mormann:2009}, \cite{Mormann:2010}, \cite{Mormann:2012}) steers close to toposes (without saying so). More explicitly, \cite{SchultzAndSpivak:2019} and \cite{FongMyersAndSpivak:2020} present a topos-theoretic ``behavioral mereology'' of systems. But there is as yet no introductory exposition of the core mereology-logic connection discussed here.

The paper proceeds as follows. \Cref{part:preliminaries} introduces the required notions of category theory. \Cref{part:toposes} turns to topos theory, showing first how a topos is a full universe of parts, and second how its internal logic --- including its truth values --- arises from the underlying structuring of those parts.
