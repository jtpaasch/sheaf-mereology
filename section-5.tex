%%%%%%%%%%%%%%%%%%%%%%%%%%%%%%%%%%%%%%%%%%
%%%%%%%%%%%%%%%%%%%%%%%%%%%%%%%%%%%%%%%%%%
%%%%%%%%%%%%%%%%%%%%%%%%%%%%%%%%%%%%%%%%%%
%%%%%%%%%%%%%%%%%%%%%%%%%%%%%%%%%%%%%%%%%%
\section{Conclusion}
\label{sec:conclusion}

\noindent
The foregoing makes clear that sheaf theory provides a rich and natural framework for modeling mereology. Because it is built around a notion of coherent gluing, it comes equipped from the outset with a notion of fusion, while parts may be understood as the local patch candidates from which those fusions are assembled. Moreover, presheaves, monopresheaves, and sheaves naturally correspond to increasingly strong gluing conditions, yielding a structured spectrum of mereological possibilities within a single mathematical setting.

The chief virtue of the framework is its separation of concerns. The algebra of regions (the base locale) captures the structural relations among parts, while the fibers encode the ontological “stuff” that inhabits those regions. Gluing is not imposed by stipulation but arises from the underlying mathematics itself, allowing fusions, persistence, overlap, and supplementation to be analyzed as structural features of the chosen presheaf. In this way, the framework is both flexible and principled: by varying the presheaf assignment or the strength of its gluing conditions, one obtains a unified family of mereological models rather than a collection of ad hoc theories.

A natural next step is to enrich this picture with modalities. In a sheaf-theoretic context, modalities arise as closure operators—also known as local operators or Lawvere–Tierney topologies. These internal modal operators provide a principled way to distinguish, for example, necessary from merely possible fusions, stable from transient fusions, or coarse-grained from fine-grained notions of parthood, further extending the expressive power of the framework without abandoning its structural foundations.

