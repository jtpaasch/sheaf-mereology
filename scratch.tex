
Standard presentations of mereology tend to take what we might call a ``parts-first'' approach. You start by taking the parthood relation as primitive, and then you proceed by stipulating axioms that govern the relation. The goal is to choose your axioms well enough that the resulting models that satisfy your theory align nicely with the actual part-whole complexes that we encounter in the world around us. 

By most standard accounts (e.g., \cite{Simons:1987}, \cite{Varzi:2011}, or \cite{CotnoirAndVarzi:2021}), partisans of the parts-first approach have more or less agreed on a common (minimal) ``core'' known as ``classical mereology.'' First, classicists adopt the following principles that govern the ordering of the parts:

\begin{itemize}

\item Parthood is reflexive, antisymmetric, and transitive (i.e., it is a partial order).

\end{itemize}

\noindent
Second, classicists adopt the following decomposition principle that governs how wholes decompose:

\begin{itemize}

\item Wholes decompose into more than one proper part (i.e., parthood obeys some form of so-called ``supplementation,'' ``remainder,'' or ``complementation'' principle which says that no whole consists of only a single proper part --- there must be some remainder or relative complement).

\end{itemize}

\noindent
Third, classicists adopt the following principle that governs how parts fuse into wholes:

\begin{itemize}

\item Any collection of parts whatever forms a fusion (i.e. unrestricted fusion).

\end{itemize}

Classicists also require (either as an explicit axiom or as a consequence of the other axioms) some version of extensionality: \todo{consult Contoir's article on this and maybe cite it}

\begin{itemize}

\item If wholes have the same parts, then they're the same wholes.

\end{itemize}

With the above axioms fixed, the classicist can then define a number of other useful notions in the obvious ways, e.g.:

\begin{itemize}

\item Overlap and underlap
\item Complement/difference
\item Etc.

\end{itemize}

At first sight, most of the classicist's principles can feel deeply intuitive. However, philosophers have objected to virtually all of them. Take for instance that parthood is transitive: if $x$ is a part of $y$ and $y$ is a part of $z$, then surely $x$ is a part of $z$. But there appear to be counter-examples. For example, my appendix is a part of me, and I am a member of the orchestra, but my appendix is not a member of the orchestra.

A standard response is to point out that this sort of objection exploits an ambiguity: we utilize different, more specialized notions of ``parthood'' when we talk about the integration of the parts of biological organisms vs. those of orchestras. My appendix is a part of me under one description (as a part of a biological organism), while I am a part of the orchestra under another (as a member of a musical ensemble). 

Defenders of the classical axioms have said that the fact that we can partition the general notion of parthood into more specialized versions only shows that the above axioms do in fact characterize a general notion of parthood, characterized precisely by the above classical notions. \addcite{Simons, Varzi, etc}

However, one can't help but feel that there is something circular about this response, since it turns on the assumption that the different notions of functional unity are species (or determinations, or partitions, or what have you) of a more generic relation. But the existence of that generic relation hasn't been established, and there is no reason to think that mereological pluralism isn't correct --- namely, that there are many parthood relations, not one. \addcite{Fine, etc.}

Another common objection to the classical approach revolves around composition principles. In particular, if we adopt unrestricted fusion, as the classical mereologist does, then we seem to get too many fusions. For instance, take the pencil on the table in front of me and your left knee. Are we really to believe that there is a fusion of that pencil and your left knee? Such a fusion would have two parts that live quite far apart (possibly even on different sides of the globe). 

Defenders of the classical approach do have a response though: just because we may not have a word or concept that names the pencil+knee fusion, that doesn't mean it doesn't exist. \addcite{Varzi, etc.} Indeed, just as Moore attempt to show that extramental things exist by holding up his two hands and saying, ``Here is one hand, here is the other,'' so too might one try to show that the pencil+knee fusion exists by saying, ``there is the pencil, there is the knee.''

Again though, one can't help but feel that there is something circular about this response. To appeal to the pencil+knee's fusion after its fused-ness has been questioned just brings the principle under scrutiny back into the mix. A better response would provide an independent reason to think that pencil+knee qualifies as more than a ``mere Cambridge'' fusion.

Whether these objections/responses constitute any real conceptual clarification or are inexorably stuck in semantic circularities is not something we want to decide here. We mention these points only because they illustrate something else: they illustrate that we seem to have intuitions not just about parts, but also about fusions. In the two objections just mentioned, we seem to have certain conceptions of integrated fusions somewhere in the back of our minds, and those seem to be driving the objections. 

For example, the reason it seems wrong to say my appendix is not part of the orchestra is because we seem to think that biological organisms are integrated in a different way than orchestras. Similarly, the reason we can so easily think that a pencil and a hand don't fuse is because they don't integrate in one of the ways that we ambiently accept as legitimate.

That leads to the following question: if we have ambient intuitions about which collections count as fusions and which ones don't, then why not take a ``fusions-first'' approach to mereology? Instead of taking parthood as the primitive relation, and then try to work up to a notion of fusions, why not take fusions as the primitive relation, and then work backwards to parts?

Such an approach is far less common in the technical mereological literature. That raises yet another question: why has the ``fusions-first'' approach been so neglected? One proposal is that it might seem to be too unwieldy. One might say that there are just too many different ways that things can integrate into wholes/fusions, and so it is a hopeless task to try and enumerate them and offer any sort of a unifying taxonomy. \addcite{Simons}.

Another reason might be that such a view would be inelegant and perhaps would even fail to qualify as an ``explanation'' of the part-whole phenomenon altogether. If you ask me to explain why various $X$s seem to exhibit the same (or sufficiently similar) properties, it would be quite dissatisfying if I said, ``that's easy, there is no unifying explanation.'' 

\todo{remove/rewrite these last two paragraphs.} The ``fusions-first'' approach is not so uncommon in the literature as I just made it out to be. Mereotopology exists as a branch of hereology, for all intensive purposes, precisely because it is the fusions-first response to the classical parts-first approach. Cite Casati and Varzi, chapter 3 and others.

Fortunately, a ``fusions-first'' approach need not be as doomed as it may seem. In this paper, we claim that there is a satisfying ``fusions-first'' approach to mereology, and we present it in what follows. To accomplish this, we will build a bridge between category theory and philosophy. In particular, we will take well-known techniques used to manage the gluing-together of parts in algebraic geometry and topos theory, and we will apply those techniques to the realm of mereology.


%%%%%%%%%%%%%%%%%%%%%%%%%%%%%%%%%%%%%%%%%%
\subsection{From Parts to Sheaves}

To begin, we want to suggest that it is useful to draw a distinction between what one might call the algebra of parts on the one hand, and the integrity or gluing-together of the parts on the other. To get a sense of what this distinction means, and why it is useful, fix a part-whole complex to analyze (a statue of Dion, let's say), and let an enumeration of its parts be given. The classical principle of unrestricted fusion says that any combination of those parts glues into a fusion. In essence, this generates all possible combinations of parts. As such, it nearly yields a complete lattice, with overlap and underlap serving as the meet and join operators. 

However, it only \emph{nearly} yields a lattice because mereologists have been relunctant to allow a bottom element. Since mereologists are ontologists, they find a null element to be ontologically suspect. And indeed, what could an empty thing that is part of all other things possibly \emph{be}? So, instead of admitting it into their mereological systems, classical mereologists have simply omitted it altogether. David Lewis (\cite{Lewis:1991}) even went so far as to formulate a version of set theory that had no empty set.

Yet despite their suspicion of a null element, classical mereologists have not shied away from allowing all possible fusions to exist, as noted already. Since the bottom element of a lattice is the empty join, we can put the point like this: classical mereologists are ontologically conservative about empty joins, and ontologically permissive about non-empty joins. As Tarski pointed out long ago, the principles of classical mereology thus yield a boolean algebra, with the bottom element removed \addcite{}.

But it is difficult to see the motivation here. On the one hand, if you want to be ontologically conservative, then why allow so many fusions? If we are going to be suspicious of an empty join, then wouldn't we also be suspicious of the fusion of (say) Dion's left hand and right knee? Conversely, if we are happy to admit the existence of entities like the fusion of Dion's left hand and right knee, then why not an empty join? 

One way to diagnose the problem is to say that we, as classical mereologists, have confused the algebra of parts with the integrity of the wholes. We have defined the algebra of parts in a combinatorial way, but then at the same time, we tried to make that algebra do ontological work. But this inevitably pulls us in two directions. So, we end up letting the algebraic aspects of our parthood relation do ontological work (creating any fusion whatever), until it goes too far (e.g. the null element), at which point we try to pull back on the ontological reigns.

For another point of tension, consider extensionality. The classicist's axiom says if $x$ and $y$ have the same parts, then $x$ = $y$. This is ontologically conservative: ``no difference without a difference maker'' \addcite{Lewis}. However, this flattens all structure, and so it judges that ``tip'' and ``pit'' cannot be different words, since they have the same parts, after flattening. But that of course feels wrong. These two words have a different ordering of letters, so why would we neglect that in determining their identity? Here too we don't want the combinatorics to do any ontological work, even though we're happy to let the join operation freely generate entities.

We can free ourselves from these sorts of tensions if we separate the algebra of parts from the integrity or gluing of the parts. Let us think of the lattice of parts merely as the abstract ``parts space,'' i.e., as the set of all \emph{possible} combinations of the given parts into larger pieces. Moreover, let us be clear that this does not do any ontological work. A ``parts space'' is just an abstract description of the various combinations of parts that could be. Think of it as a kind of mold that has slots that could be filled in with actual pieces. To specify an \emph{actual} part-whole complex that occupies that parts space, we need to take a second step and fill in certain of those slots with actual stuff, and specify which of those pieces glue together into bigger pieces. 

Once we have made this distinction, we can let the algebra of parts be an algebra, and we can even allow a bottom element without worry. As a component of the abstract parts space, the bottom element is no more a real thing than the join of any other arbitrary regions of the parts space. At the same time, when we specify which pieces really occupy the parts space, we can be as ontologically conservative or as permissive as we like. We have the freedom to provide gluing conditions that are as fine-grained as we need. For instance, we can say that certain pieces glue together, while others do not (e.g., Dion's right knee glues directly to his right femur, but not directly to his left hand). Moreover, we can let the identity conditions be determined by the gluing conditions, and so maintain structured extensionality (a difference in fusions comes from different parts, or different gluings).

Once we make the distinction between the background algebra of parts and the foreground integrity of the fusion, our task takes on a distinctive shape: now we find ourselves trying to coherently glue pieces together over an ambient space. And that is something known well to algebraic geometers and topos theorists: it is the task of constructing a sheaf over a space. For the algebraic geometer and topos theorist, sheaf theory provides a systematic framework for gluing together pieces over a space in such a way that the gluing is done coherently and consistently against the ambient structure of the underlying space. It stands to reason, then, that the mathematician's sheaf-theoretic techniques can be used profitably in mereology.


%%%%%%%%%%%%%%%%%%%%%%%%%%%%%%%%%%%%%%%%%%
\subsection{The Central Thesis}

In this paper, we want to build a bridge between sheaf theory and mereology by importing some of those sheaf-theoretic techniques into the mereological setting. The central claim of this paper is thus: part-whole complexes can be usefully modeled as sheaves over locales. The key ideas are as follows:

\begin{itemize}

\item An algebra of parts tells us all the ways that parts can combine to form bigger wholes. In this sense, an algebra of parts generates the ambient ``parts space'' of an object, i.e. the abstract lattice-theoretic structure of \emph{possible} combinations. But not all possible combinations actually glue together to form \emph{actual} fusions. In many cases, we want to allow that only some of the parts glue together into an integrated whole/fusion. 

\item So, we then require a separate step where we, the mereologists, have to ``fill in'' the abstract parts space with actual parts: we have to specify which bits of stuff fill in or occupy which slots in that ambient parts space, and we have to specify how those various bits of stuff glue together to form integrated fusions.

\item It is tempting to try to model the ambient parts space as a topology. However, topologies have points, and it is not clear that all of the part-whole complexes that we might wish to consider are usefully modeled with points. This limitation is easy to overcome if we generalize and move to the point-free setting: instead of a topology, we choose to model the ambient parts space as a locale (a point-free generalization of a topology). 

\item We use a sheaf to specify which bits of stuff inhabit an ambient locale and also to stipulate how those bits glue together. A sheaf is precisely an assignment of data to a topology or locale that coherently glues that data together. So, we choose to model the actual part-whole complex as a sheaf over the ambient locale. 

\item To specify a part-whole complex, then, we (the mereologists) simply need to define a sheaf over the given locale. The ``data'' that we assign to the ambient locale are the bits of actual stuff that inhabit that parts space, and the gluing condition specifies how those pieces glue together.

\item This yields a straightforward procedure that can be used to model any part-whole complex: first, specify the ambient locale, i.e., the abstract space of parts that the part-whole complex in question inhabits; second, fill in that ambient space with actual pieces and say how they glue together; third, let the sheaf framework do the rest of the work. Then the glued sections of the sheaf turn out to be the fusions, whose parts are the smaller sections each fusion is glued from. This is an honest ``fusions-first'' approach. 

\end{itemize}

\noindent
There are two important benefits that come along for free when we take this approach.

\begin{itemize}

\item The sheaves over a locale form a topos. A topos is a special kind of category that you can do ``parts''-like logic in. Indeed, every topos comes equipped with just such an internal logic. It turns out that this internal logic corresponds exactly to the correct mereological logic that governs the part-whole complexes that can be formed over that locale. So, there is no need to manually create a mereological logic to reason about the part-whole complexes that occupy the ambient locale. We get that for free. 

\item Modalities are natural operators that occur in sheaves, where they are easily defined and managed. These modalities interact correctly with the internal logic of the topos (and in fact are part of that internal logic). So we get mereological modalities for free too.

\end{itemize}

To our mind, the fact that these benefits come for free offers a compelling reason to adopt a sheaf-theoretic approach to part-whole complexes.


%%%%%%%%%%%%%%%%%%%%%%%%%%%%%%%%%%%%%%%%%%
\subsection{Literature}

The literature on mereology is vast. What we might think of as formal mereology (axiomatized systems) goes back at least to Le\'{s}niewski's system called ``Mereology'' [cite] and Leonard and Goodman's ``Calculus of Individuals'' (\cite{LeonardAndGoodman:1940} and \cite{Goodman:1977}), along with other contributions by Whitehead [cite], Tarski (\cite{Tarski:1986} and \cite{Tarski:1956}), Rescher (\cite{Rescher:1955}), and others. Surveys of the resulting literature and ideas can be found in the now standard works by Simons and others (e.g. \cite{Eberle:1970}, \cite{Simons:1987}, \cite{Varzi:2011}, or \cite{CotnoirAndVarzi:2021}).

Topological concepts have been used in mereology for a long time (see the historical coverage and discussion in \cite{Simons:1987}). So-called ``mereotopology'' explicitly aims to characterize mereological questions in topological ways, especially using notions like boundaries, interiors, and connectedness. A standard introduction to modern mereotopology is \cite{CasatiAndVarzi:1999}.

However, despite its heavy reliance on topology, mereotopology has not (to our knowledge) utilized sheaf-theory in any significant way (nor has classical mereology). \todo{discuss Spivak's behavioral mereology/temporal type theory} (\cite{FongMyersAndSpivak:2020}, \cite{SchultzAndSpivak:2019}), Moltmann's trope sheaves, Moltmann's mereology.

For sheaves, see \cite{Tennison:1975}, \cite{MacLaneAndMoerdijk:1994}, \cite{Rosiak:2022}, or \cite{Wedhorn:2016}. For locales, see \cite{Johnstone:1982}, \cite{Vickers:1989}, \cite{Vickers:2007}, \cite{PicadoAndPultr:2012}.

For toposes, see \cite{Goldblatt:1984}, \cite{McLarty:1992}, \cite{Borceux:1994}, \cite{MacLaneAndMoerdijk:1994}, \cite{LawvereAndSchanuel:1997}. \todo{discuss the idea of ``deriving'' the logic from the underlying structure, rather than ``inventing'' it axiomatically. Perhaps Moltmann's mereology is to be cited here.}

\todo{discuss how mereological concepts are used in certain ``fusion-first'' approaches, e.g., Peter Simons and using ``connectedness.'' Discuss Van Inwagen's special composition question (\cite{VanInwagen:1990}).}

\todo{discuss non-boolean approaches. Discuss boolean algebra stuff from Protow, the survey ``Logic in Heyting Algebras,''  Moltmann's Heyting mereology.}


%%%%%%%%%%%%%%%%%%%%%%%%%%%%%%%%%%%%%%%%%%
\subsection{Contributions}

The contributions of this paper are as follows:

\begin{itemize}

\item We demonstrate a viable ``fusions-first'' approach to mereology.
\item We separate the algebra of parts from the integrity of fusions.
\item We build a bridge between mereological techniques of mathematics and philosophy. In particular:

\begin{itemize}

\item We utilize sheaves to systematically manage coherence and gluing over parts spaces.
\item We acquire the correct mereological logics for free from the internal language of the underlying topos.

\end{itemize}

\end{itemize}


%%%%%%%%%%%%%%%%%%%%%%%%%%%%%%%%%%%%%%%%%%
\subsection{Plan of the Paper}

The plan of this paper is as follows. 

\begin{itemize}

\item In \cref{sec:sheaf-theory}, we introduce the relevant parts of sheaf theory that will be used in the rest of the paper.

\item In \cref{sec:sheaf-mereology}, we define part and whole in sheaf-theoretic terms, and we show how to model different kinds of part-whole complexes as sheaves. 

\item In \cref{sec:modalities}, we show how mereological modalities arise naturally in sheaves.

\item In \cref{sec:classical-mereology-in-sheaves}, we discuss what classical mereological notions look like in the sheaf-theoretic setting.

end{itemize}



%%%%%%%%%%%%%%%%%%%%%%%%%%%%%%%%%%%%%%%%%%
%%%%%%%%%%%%%%%%%%%%%%%%%%%%%%%%%%%%%%%%%%
%%%%%%%%%%%%%%%%%%%%%%%%%%%%%%%%%%%%%%%%%%
%%%%%%%%%%%%%%%%%%%%%%%%%%%%%%%%%%%%%%%%%%
\section{Classical Mereological Notions in the Sheaf-theoretic Setting}
\label{sec:classical-mereology-in-sheaves}

\noindent
In this section, we provide a discussion of what classical notions of mereology look like in the sheaf-theoretic setting.

\begin{itemize}

\item \emph{Cambridge fusions}. Sheaves handle Cambridge fusions correctly.

\item \emph{Mere collections}. The collection of all dogs. Is that a ``whole''? Well, we could build a sheaf whose atomic regions are filled with dogs, none of which glue. Then we have a collection of dogs, but no glued object. That matches exactly the intuition: yes, we have a ``collection'' (we built a sheaf for it, after all), but the internals of that sheaf reveal that it's \emph{merely} a collection, i.e., that its parts are not glued.

\item \emph{Co-habitating fusions}. Sheaves allow multiple fusions to occupy the same locale, without being glued. For instance, in the sheaf of real-valued functions over real number line, there are many functions that glue together, and occupy the same locale. 

\item \emph{Non-boolean algebra}. The parts space is Heyting, not Boolean. We're not saddled with such a strong complement operation. You can pick a locale that is Boolean if you need it, but this framework doesn't require it. In fact, the positive logic of a locale is ``geometric logic.''

\item \emph{Reflexivity, antisymmetry, and transitivity}. These are guaranteed. Locally, of course, you may not have transitivity. But globally, it's a theorem. [Check that.]

\item \emph{Distributivity}. \todo{do the glued sections of a sheaf have to be distrubitive? Only inside what glues (since we glue pairwise, so every $i \join/ j$ of the cover.}

\item \emph{An empty element}. There is a need for a bottom element in the \emph{algebra} of parts, but a sheaf need not contain any such thing. There is no need here to try and construct awkward mathematical structures that do algebra on parts but yet don't have a bottom element because our ontological intuitions tell us there can be no such thing. That confuses two issues: algebra and integrity. So here we separate those cleanly, and the algebra can do algebra while the sheaf can do integrity. [In a sheaf you CAN'T assign an empty element to bottom, for coherence, so the bottom element is special...need to say more about that and figure it out.]

\item \emph{Supplementation principles}. Sheaves don't constrain one way or another. [Is that really true? Maybe it's better to say that it doesn't force any supplementation principles, which might provide a reason to call into question whether supplementation is another one of those ideas that is about integrity of parts but has been confused with the algebra of parts.]

\item \emph{Ordering of parts}. Consider that ``pit'' and ``tip'' have the same parts but are different words. These differences can be handled by different sheaves over a 3-stage prefix-ordered locale as in the example of concurrent processes. Note that we retain extensionality.

\item \emph{Extensionality}. Classical mereology's notion of extensionality essentially flattens any structure and is thus overly aggressive. This is why extensionality is so controversial. The sheaf-theoretic perspective retains extensionality, but is much more nuanced. [Here too, I suspect that mereological discussions of extensionality have confused the algebra of parts and the integrity of wholes.]

\item \emph{Gunk and atoms}. You can model continuity and gunky parts if you so desire. You just need a sober space to do it. \todo{check that we can model continuity in the locale in this way.} \todo{can you do continuity only in the sheaf data, without an underlying continuous decompositon in the locale? I would think that if you can't infinitely decompose into smaller opens around a point in the locale, you couldn't do such a thing in the sheaf data?}

\item \emph{Priority of wholes}. The framework is agnostic as to whether you take an  Aristotelian-Thomistic approach\addcite{Aquinas, Arlig, and that guy who wrote that recent book defending the Aristotelian view}.

\item \emph{The whole is greater than its parts}. The framework is agnostic as to whether you want to be a Scotist and say that the whole is something over and above its parts (cite Cross) or an Ockhamist who says it is not\addcite{Normore, Arlig}.

\end{itemize}



%%%%%%%%%%%%%%%%%%%%%%%%%%%%%%%%%%%%%%%%%%
%%%%%%%%%%%%%%%%%%%%%%%%%%%%%%%%%%%%%%%%%%
\subsection{Supplementation}
\label{sec:supplementation}

\noindent
Supplementation is the idea that fusions are not made from a single proper part. If you remove a proper part from a fusion, there should be at least one other proper part left over.

There are weaker and stronger formulations. In this setting, weak supplementation is the claim that if $s$ is a proper part of $t$, then $t$ has another (proper) part $r$ disjoint from $s$. 

\begin{Definition}[Weak (regional) supplementation]

We say that weak (mereological) supplementation holds in a presheaf $F$ iff, for any $s \in F(V)$ and $t \in F(U)$:

\[
  s \properPartOf/ t \implies \exists r \in F(W) (r \properPartOf/ t \text{ and } \lnot(r \partOverlap/ s)).
\]

\end{Definition}

\noindent
In other words, every proper part has another proper part beside it, as it were.

Note that disjointness of fusions $r$ and $s$ implies disjointness of regions $W$ and $V$, while $s \properPartOf/ t$ means that $V \strictChildOf/ U$. Thus, for weak supplementation to hold, there must be another region $W$ disjoint from $V$. This is purely a requirement on the available regions.

\begin{Definition}[Local regional supplementation]

We say that a locale $\category{L}$ is locally regionally supplemented iff, for all $U \in \category{L}$ and all $V \strictChildOf/ U$ with $V \not = \bottom/$, there exists a $W \not = \bottom/$ such that

\[
  W \strictChildOf/ U
  \quad \text{ and } \quad
  W \meet/ V = \bottom/.
\]

\end{Definition}

\noindent
In other words, every proper subregion of $U$ has another, disjoint proper subregion of $U$ beside it, as it were.

This is local (to $U$) because it only requires that the regions in $U$'s downset are supplemented. It makes no claim that regions outside of $U$'s downset are supplemented.

Boolean locales are locally weakly supplemented. 

\begin{Theorem}[Boolean locales are regionally supplemented]

Let $\category{L}$ be a locale. If $\category{L}$ is Boolean, then $\category{L}$ is locally regionally supplemented.

\end{Theorem}

\begin{proof}

Suppose $\category{L}$ is Boolean, and fix a $V \strictChildOf/ U$ with $V \not = \bottom/$. Since $\category{L}$ is Boolean, $V$ has a complement $\lnot V$ satisfying

\[
  V \meet/ \lnot V = \bottom/
  \quad \text{ and } \quad
  V \join/ \lnot V = \top.
\]

\noindent
Define $W = U \meet/ \lnot V$. Then:

\begin{enumerate}
  \item 
    In a locale, $a \meet/ b \childOf/ a$. Let $a = U$ and $b = \lnot V$. Then
    $U \meet/ \lnot V \childOf/ U$. Substituting $W$ yields $W \childOf/ U$.
  \item 
    $W \meet/ V$ = $(U \meet/ \lnot V) \meet/ V$ = 
    $U \meet/ (\lnot V \meet/ V)$ = $U \meet/ \bottom/ = \bottom/$.
  \item
    In a Boolean locale, $a \meet/ b = \bottom/ \Longleftrightarrow a \childOf/ \lnot b$. 
    If we assume for contradiction that $U \meet/ \lnot V = \bottom/$, it therefore follows that
    $U \childOf/ \lnot(\lnot V)$. But since $\lnot \lnot a = a$ in a Boolean locale, 
    $U \childOf/ V$. That contradicts the assumption $V \strictChildOf/ U$.
    Hence, $W = U \meet/ \lnot V \lnot = \bottom/$.
\end{enumerate}

\noindent
Thus, $W$ witnesses local regional supplementation.
\end{proof}

\noindent
Although Boolean locales are locally regionally supplemented, it does not go the other way.

\begin{Theorem}[Local regional supplementation need not be Boolean]

Let $\category{L}$ be a locale. It is not the case that if $\category{L}$ is locally regionally supplemented then $\category{L}$ is Boolean.

\end{Theorem}

\begin{proof}

Take a connected topological Hausdorff space $X$. Since it is connected, it cannot be partitioned into two disjoint non-empty opens, i.e., there do not exist two disjoint, nonempty opens $A, B \subseteq X$ such that $X = A \cup B$. 

Fix an open $\EmptySet/ \subset U \subset X$, and define $\lnot U$ as the largest region disjoint from $U$:

\[
  \lnot U = \bigjoin/ \{ V \mid V \cap U = \EmptySet/ \}.
\]

\noindent
For any subset $D \subseteq X$, the interior $\e{int}(D)$ is the largest open of $D$. But any open disjoint from $U$ must be contained in $X - U$, and the join of all of those is the largest such region. So $\lnot U$ is the interior of $X - U$. We thus have:

\[
  U \cup \lnot U = U \cup \e{int}(X - U).
\]

\noindent
Assume now, for contradiction, that $U \cup \e{int}(X - U) = X$. But notice that $U$ and $\e{int}(X - U)$ are disjoint, since $\e{int}(X - U) \subseteq X - U$.  Further, $U$ is non-empty because we fixed $\EmptySet/ \subset U$, and $\e{int}(X - U)$ is also non-empty because we assumed that $U \subset X$ and $U \cup \e{int}(X - U) = X$. But then $X$ is partitioned into two disjoint non-empty opens, namely $U$ and $\e{int}(X - U)$, which contradicts connectedness. Thus it cannot be that $U \cup \e{int}(X - U) = X$. Since $\e{int}(X - U) = \lnot U$, this means $U \cup \lnot U \not = X$. Hence, $U \cup \lnot U$ is a strict subset of $X$, i.e., $U \cup \lnot U \subset X$. 

If $X$ were Boolean, then $U \cup \lnot U$ would equal $X$. Since that is not true here, we have shown that the topology on $X$ is not Boolean.

Nevertheless, the topology on $X$ is locally regionally supplemented. Pick any $V \subset U$, and write $\bar{V}$ for the closure of $V$. Since $V$ is a proper subset of $U$, there is a point $x \in U - \bar{V}$. Pick a small neighborhood $W$ around $x$ that is strictly contained in $U$ but disjoint from $V$. This $W$ witnesses local regional supplementation: it is not empty, it is strictly contained in $U$, and it is disjoint from $V$.

Since $X$ is locally regionally supplemented but not Boolean, the locale $\category{L}$ obtained by taking the poset of opens of $X$ is too. Hence, there exist connected locales that are regionally supplemented but not Boolean. 
\end{proof}

Now consider the strong form of mereological supplementation.

\begin{Definition}[Strong (mereological) supplementation]

We say that strong (mereological) supplementation holds in a presheaf $F$ iff, for any fusions $s \in F(V)$ and $t \in F(U)$:

\[
  s \not \partOf/ t \implies \exists r \in F(W) (r \properPartOf/ t \text{ and } \lnot(r \partOverlap/ s)).
\]

\end{Definition}

\noindent
In other words, if $s$ is outside of $t$, it is disjoint from some part of $t$. 

Notice that $s \not \partOf/ t$ amounts to $V \not \childOf/ U$, and $r \properPartOf/ t$ amounts to $W \strictChildOf/ U$, and $\lnot(r \partOverlap/ s)$ means $W \meet/ V = \bottom/$. So this too reduces to a purely regional requirement.

\begin{Definition}[Global regional supplementation]

We say that a locale $\category{L}$ is globally regionally supplemented iff, for all $V, U \in \category{L}$ such that $V \not \childOf/ U$ and $U \not = \bottom/$, there exists a $W \not = \bottom/$ such that

\[
  W \strictChildOf/ U
  \quad \text{ and } \quad
  W \meet/ V = \bottom/.
\]

\end{Definition}

\noindent
In other words, any region outside of $U$ is disjoint from some subregion inside $U$.

As we saw, local regional supplementation is too weak of a requirement to guarantee Booleanness. But global regional supplementation is different: locales that are globally regionally supplemented are Boolean.

\begin{Theorem}[Global regional supplementation is Boolean]

Let $\category{L}$ be a locale. If $\category{L}$ is globally regionally supplemented then $\category{L}$ is Boolean.

\end{Theorem}

\begin{proof}

Assume $\category{L}$ is globally regionally supplemented, and fix a $U$. Collect all regions disjoin from $U$:

\[
  \mathcal D_{U} = \{ X \mid X \meet/ U = \bottom/ \}
\]

\noindent
Define the complement as the largest region disjoint from $U$:

\[
  U = \bigjoin/ \mathcal D_{U}.
\]

\noindent
This join always exists because $\category{L}$ is a complete lattice.  

$U \meet/ \lnot U = \bottom/$ by distributivity:

\[
 U \meet/ \lnot U = 
   U \meet/ \bigjoin/_{X \in \mathcal D_{U}}X = 
   \bigjoin/_{X \in \mathcal D_{U}} (U \meet/ X) =
   \bigjoin/_{X \in \mathcal D_{U}} (\bottom/) = \bottom/.
\]

\noindent
Next, assume for contradiction that $U \join/ \lnot U \strictChildOf/ \top$. Thus, $U \not = \top$. Let $V = \top$, so that $V \not \strictChildOf/ U$ and the antecedent of global regional supplementation is satisfied. By global regional supplementation, there exists a region $W \strictChildOf/ U$ disjoint from $V$ (so $W \meet/ V = \bottom/$). Since $V = \top$, $W \meet/ \top = W = \bottom/$. But global regional supplementation says that $W$ is not $\bottom/$, so this is a contradiction. Hence, it cannot be that $U \join/ \lnot U$ is a strict sub-region of $\top$. Rather, it must be that $U \join/ \lnot U = \top$. 

Since $U$ is arbitrary while $U \meet/ \lnot U = \bottom/$ and $U \join/ \lnot U = \top$, we have shown that every $U$ has a Boolean complement, namely $\lnot U$. Thus $\category{L}$ is Boolean.
\end{proof}

To summarize, presheaves need not satisfy weak supplementation. In the sheaf-theoretic setting, weak (mereological) supplementation reduces to a local regional supplementation, but this is a fairly weak requirement, and non-Boolean locales exist that satisfy it. Strong (mereological) supplementation reduces to global regional supplementation, and this is a strong property: any locale that satisfies it is Boolean. 

\begin{Remark}

Since weak and strong (mereological) supplementation reduces to purely regional requirements, these results apply equally to presheaves, monopresheaves, and sheaves, since supplementation is a property of the underlying locales. Supplementation is a geometric property of the parts space, not the parts themselves.

\end{Remark}
