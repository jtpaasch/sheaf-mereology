%%%%%%%%%%%%%%%%%%%%%%%%%%%%%%%%%%%%%%%%%%
%%%%%%%%%%%%%%%%%%%%%%%%%%%%%%%%%%%%%%%%%%
%%%%%%%%%%%%%%%%%%%%%%%%%%%%%%%%%%%%%%%%%%
%%%%%%%%%%%%%%%%%%%%%%%%%%%%%%%%%%%%%%%%%%
\section{Sheaf Theory}
\label{sec:sheaf-theory}

In this section, we introduce the parts of sheaf theory needed for the sequel.


%%%%%%%%%%%%%%%%%%%%%%%%%%%%%%%%%%%%%%%%%%
\subsection{Fibers}

\noindent
Suppose we have a map (function) $f: E \to B$ that looks something like this:

\begin{diagram}

\node at (-5, 0) {$E$};
\draw (-3, -0.15) ellipse (1.5cm and 1.85cm);
\node[dot, label=left:$1$] (1) at (-3.25, 1.25) {};
\node[dot, label=left:$2$] (2) at (-2.5, 0.75) {};
\node[dot, label=left:$3$] (3) at (-3, 0.25) {};
\node[dot, label=left:$4$] (4) at (-3.5, -0.25) {};
\node[dot, label=left:$5$] (5) at (-2.5, -0.5) {};
\node[dot, label=left:$6$] (6) at (-3.0, -1) {};
\node[dot, label=left:$7$] (7) at (-2.75, -1.5) {};

\node at (4.375, 0) {$B$};
\draw (3.15, 0) ellipse (0.75cm and 1.75cm);
\node[dot, label=right:$a$] (a) at (3, 1) {};
\node[dot, label=right:$b$] (b) at (3, 0) {};
\node[dot, label=right:$c$] (c) at (3, -1) {};

\node at (0, -1.625) {$f$};
\draw[arrow,->] (1) to (a);
\draw[arrow,->] (2) to (b);
\draw[arrow,->] (3) to (a);
\draw[arrow,->] (4) to (b);
\draw[arrow,->] (5) to (c);
\draw[arrow,->] (6) to (b);
\draw[arrow,->] (7) to (c);

\end{diagram}

It is sometimes convenient to turn the diagram sideways and group together points in the domain that get sent to the same target, like so:

\begin{diagram}

\node at (3, 1.85) {$E$};
\draw (0, 1.85) ellipse (2.5cm and 1cm);
\node[dot, label=above:$1$] (1) at (-2, 1.75) {};
\node[dot, label=above:$3$] (3) at (-1.5, 1.5) {};
\node[dot, label=above:$2$] (2) at (-0.5, 1.75) {};
\node[dot, label=above:$4$] (4) at (0, 2) {};
\node[dot, label=above:$6$] (6) at (0.5, 1.55) {};
\node[dot, label=above:$5$] (5) at (2, 1.75) {};
\node[dot, label=above:$7$] (7) at (1.5, 1.5) {};

\node at (3, 0) {$B$};
\draw (0, -0.15) ellipse (2.5cm and 0.75cm);
\node[dot, label=below:$a$] (a) at (-1.75, 0) {};
\node[dot, label=below:$b$] (b) at (0, 0) {};
\node[dot, label=below:$c$] (c) at (1.75, 0) {};

\node at (2.5, 0.875) {$f$};
\draw[arrow,->] (1) to (a);
\draw[arrow,->] (2) to (b);
\draw[arrow,->] (3) to (a);
\draw[arrow,->] (4) to (b);
\draw[arrow,->] (5) to (c);
\draw[arrow,->] (6) to (b);
\draw[arrow,->] (7) to (c);

\end{diagram}

This presentation makes the pre-images of elements in $B$ very easy to see. For any point in $B$, its pre-image is just the set of points sitting ``over'' it:

\begin{diagram}

\node at (3, 1.85) {$E$};

\draw (-1.75, 1.85) ellipse (0.7cm and 0.75cm);
\node[dot, label=above:$1$] (1) at (-2, 1.75) {};
\node[dot, label=above:$3$] (3) at (-1.5, 1.5) {};

\draw (0, 1.95) ellipse (0.85cm and 0.85cm);
\node[dot, label=above:$2$] (2) at (-0.5, 1.75) {};
\node[dot, label=above:$4$] (4) at (0, 2) {};
\node[dot, label=above:$6$] (6) at (0.5, 1.55) {};

\draw (1.75, 1.85) ellipse (0.7cm and 0.75cm);
\node[dot, label=above:$5$] (5) at (2, 1.75) {};
\node[dot, label=above:$7$] (7) at (1.5, 1.5) {};

\node at (3, 0) {$B$};
\draw (0, -0.15) ellipse (2.5cm and 0.75cm);
\node[dot, label=below:$a$] (a) at (-1.75, 0) {};
\node[dot, label=below:$b$] (b) at (0, 0) {};
\node[dot, label=below:$c$] (c) at (1.75, 0) {};

\node at (2.5, 0.875) {$f$};
\draw[arrow,->] (1) to (a);
\draw[arrow,->] (2) to (b);
\draw[arrow,->] (3) to (a);
\draw[arrow,->] (4) to (b);
\draw[arrow,->] (5) to (c);
\draw[arrow,->] (6) to (b);
\draw[arrow,->] (7) to (c);

\end{diagram}

If we stack the points in each pre-image vertically, one on top of the other, we can then think of each pre-image as a kind of ``stalk'' growing over its base point:

\begin{diagram}

\node at (2.825, 1.5) {$E$};
\draw (-1.75, 1.75) ellipse (0.5cm and 1cm);
\node[dot, label=above:$3$] (3) at (-1.75, 2) {};
\node[dot, label=above:$1$] (1) at (-1.75, 1.25) {};

\draw (0, 2.15) ellipse (0.5cm and 1.375cm);
\node[dot, label=above:$6$] (6) at (0, 2.75) {};
\node[dot, label=above:$4$] (4) at (0, 2) {};
\node[dot, label=above:$2$] (2) at (0, 1.25) {};

\draw (1.75, 1.75) ellipse (0.5cm and 1cm);
\node[dot, label=above:$7$] (7) at (1.75, 2) {};
\node[dot, label=above:$5$] (5) at (1.75, 1.25) {};

\node at (2.825, -0.15) {$B$};
\draw (0, -0.15) ellipse (2.5cm and 0.75cm);
\node[dot, label=below:$a$] (a) at (-1.75, 0) {};
\node[dot, label=below:$b$] (b) at (0, 0) {};
\node[dot, label=below:$c$] (c) at (1.75, 0) {};

\node at (2.825, 0.625) {$f$};
\draw[arrow,->] (-1.75, 1) to (a);
\draw[arrow,->] (0, 1) to (b);
\draw[arrow,->] (1.75, 1) to (c);

\end{diagram}

This gives rise to the idea of the ``fibers'' of a map. The fibers of a map are just its pre-images. For instance, the fiber of $f$ over $b$ is $\{ 2, 4, 6 \}$.

% ----------------------------------------
\begin{Definition}[Fiber]

Given a map $f : E \to B$ and a point $y \in B$, the \emph{fiber} over $y$ is its pre-image $\preimage{f}(y)$ = $\{ x \mid f(x) = y \}$. $B$ is called the \emph{base space} of $f$, and $y$ the \emph{base poin}t of the fiber.

\end{Definition}

We can take a cross-section of one or more fibers by selecting a point from each of the fibers in question. For instance, we can take $3$, $4$, and $7$ as a cross-section of the fibers $\preimage{f}(a)$, $\preimage{f}(b)$, and $\preimage{f}(c)$:

\begin{diagram}

\draw[rounded corners=4pt,fill=selected] (-2.75, 2.15) rectangle (2.75, 1.875);

\node at (2.825, 1.5) {$E$};
\draw (-1.75, 1.75) ellipse (0.5cm and 1cm);
\node[dot, label=above:$3$] (3) at (-1.75, 2) {};
\node[dot, label=above:$1$] (1) at (-1.75, 1.25) {};

\draw (0, 2.15) ellipse (0.5cm and 1.375cm);
\node[dot, label=above:$6$] (6) at (0, 2.75) {};
\node[dot, label=above:$4$] (4) at (0, 2) {};
\node[dot, label=above:$2$] (2) at (0, 1.25) {};

\draw (1.75, 1.75) ellipse (0.5cm and 1cm);
\node[dot, label=above:$7$] (7) at (1.75, 2) {};
\node[dot, label=above:$5$] (5) at (1.75, 1.25) {};

\node at (2.825, -0.15) {$B$};
\draw (0, -0.15) ellipse (2.5cm and 0.75cm);
\node[dot, label=below:$a$] (a) at (-1.75, 0) {};
\node[dot, label=below:$b$] (b) at (0, 0) {};
\node[dot, label=below:$c$] (c) at (1.75, 0) {};

\node at (2.825, 0.625) {$f$};
\draw[arrow,->] (-1.75, 1) to (a);
\draw[arrow,->] (0, 1) to (b);
\draw[arrow,->] (1.75, 1) to (c);

\end{diagram}

We can also take cross-sections local to only some of the fibers. For instance, we can take $1$ and $4$ as a cross-section of $\preimage{f}(a)$ and $\preimage{f}(b)$:

\begin{diagram}

\draw[rounded corners=4pt,fill=selected] (-2.5, 1.05) -- (0.7, 2.45) -- (0.8, 2.2) -- (-2.4, 0.8) -- cycle;

\node at (2.825, 1.5) {$E$};
\draw (-1.75, 1.75) ellipse (0.5cm and 1cm);
\node[dot, label=above:$3$] (3) at (-1.75, 2) {};
\node[dot, label=above:$1$] (1) at (-1.75, 1.25) {};

\draw (0, 2.15) ellipse (0.5cm and 1.375cm);
\node[dot, label=above:$6$] (6) at (0, 2.75) {};
\node[dot, label=above:$4$] (4) at (0, 2) {};
\node[dot, label=above:$2$] (2) at (0, 1.25) {};

\draw (1.75, 1.75) ellipse (0.5cm and 1cm);
\node[dot, label=above:$7$] (7) at (1.75, 2) {};
\node[dot, label=above:$5$] (5) at (1.75, 1.25) {};

\node at (2.825, -0.15) {$B$};
\draw (0, -0.15) ellipse (2.5cm and 0.75cm);
\node[dot, label=below:$a$] (a) at (-1.75, 0) {};
\node[dot, label=below:$b$] (b) at (0, 0) {};
\node[dot, label=below:$c$] (c) at (1.75, 0) {};

\node at (2.825, 0.625) {$f$};
\draw[arrow,->] (-1.75, 1) to (a);
\draw[arrow,->] (0, 1) to (b);
\draw[arrow,->] (1.75, 1) to (c);

\end{diagram}

% ----------------------------------------
\begin{Definition}[Section]

Given a map $f : E \to B$ and a subset of base points $C \subseteq B$, a \emph{section} of $f$ (over $C$) is a choice of one element from each fiber over each base point $x \in C$.

\end{Definition}

% ----------------------------------------
\begin{Remark}
\label{remark:section-terminology}

Since each element in a fiber amounts to a section over the fiber's base point, the elements of a fiber are often just called the sections of the fiber.

\end{Remark}


%%%%%%%%%%%%%%%%%%%%%%%%%%%%%%%%%%%%%%%%%
\subsection{Spaces}

\noindent
In the above examples, the base $B$ was a set. We often want to consider bases that have more structure, e.g., bases that have spatial structure.

In traditional topology, spaces are built out of the points of the space. Given a set of points, a topology on that set specifies which points belong in which regions of the space.

% ----------------------------------------
\begin{Definition}[Topology]

Let $X$ be a non-empty set, thought of as the points of the space. A \emph{topology} on X is a collection $T$ of subsets of $X$, thought of as the regions of the space and called the \emph{open sets} (or just the \emph{opens}) of $T$, that satisfy the following conditions:

\begin{enumerate}

\item [(O1)] The empty set and the whole set are open:

$$\EmptySet/ \in T, X \in T.$$

\item [(O2)] Arbitrary unions of opens are open:

$$\text{if } \{ U_{i} \}_{i \in I} \subseteq T, \text{ then } \bigcup\limits_{i \in I} U_{i} \in T.$$

\item [(O3)] Finite intersections of opens are open:

$$\text{if } U_{1}, \ldots, U_{n} \in T, \text{ then } \bigcap\limits_{i=1}^{n} U_{i} \in T.$$

\end{enumerate} 

\end{Definition}

These conditions encode the way that spatial regions are put together. For instance, it ensures that if two regions overlap, then their overlapping area is a region too.

\begin{Remark}

The regions of a topology, ordered by inclusion, form a complete lattice. Since the topology includes arbitrary unions, the join of this lattice is set union, but since the topology includes only finite intersections, the meet of this lattice is the \emph{interior} of set intersection.

\end{Remark}


% ----------------------------------------
\begin{Example}
\label{ex:topology}

Let $X = \{ a, b, c \}$. One possible topology is $T = \{ \EmptySet/, \{ b \}, \{ a, b \}, \{ b, c \}, \{a, b, c \} \}$. If we draw dashed circles around each of the opens (regions), we get:

\begin{diagram}

\node[dot, label=below:$a$] (a) at (-1.75, 0) {};
\node[dot, label=below:$b$] (b) at (0, 0) {};
\node[dot, label=below:$c$] (c) at (1.75, 0) {};

\draw[dashed] (0, 0) ellipse (2.5cm and 1.55cm);
\draw[dashed] (0, -0.1) ellipse (0.5cm and 0.5cm);
\draw[dashed] (-0.75, -0.1) ellipse (1.55cm and 1cm);
\draw[dashed] (0.75, -0.1) ellipse (1.55cm and 1cm);

\end{diagram}

There are two regions $\{ a, b \}$ and $\{ b, c \}$ that overlap at $b$ (so $\{ b \}$ is a region in $T$ too). There is also the full region $\{ a, b, c \}$, which is the union of the smaller regions.

We can draw $T$ as a Hasse diagram, which shows that the regions form a lattice:

\begin{diagram}

\node (abc) at (0, 3) {$\{ a, b, c \}$};
\node (ab) at (-1, 2) {$\{ a, b \}$};
\node (bc) at (1, 2) {$\{ b, c \}$};
\node (b) at (0, 1) {$\{ b \}$};
\node (emptyset) at (0, 0) {$\EmptySet/$};

\draw (emptyset) to (b);
\draw (b) to (ab);
\draw (b) to (bc);
\draw (ab) to (abc);
\draw (bc) to (abc);

\end{diagram}

\end{Example}

The lattice structure suggests that much of what is important about a space is not so much its points, but rather its opens/regions. This leads to the idea that topology-like reasoning can be done without the points. So, we can generalize: take a topology, and drop the points. That leaves just the opens/regions, which we call a frame (or locale).


% ----------------------------------------
\begin{Definition}[Frame/Locale]

A \emph{frame} (synonymously, a \emph{locale}) $\category{L}$ is a partially ordered set $L$ (whose elements are called opens or regions) that satisfies the following conditions:

\begin{enumerate}

\item [(L1)] $L$ is a complete lattice:

  \begin{itemize}
  \item Every subset $S \subseteq L$ has a join, denoted $\bigjoin/ S.$
  \item Every finite subset $F \subseteq L$ has a meet, denoted $\bigmeet/ F.$
  \end{itemize}

\item [(L2)] Finite meets distribute over arbitrary joins:
$$
  a \meet/ \bigjoin/\limits_{i \in I} b_{i} = \bigjoin/\limits_{i \in I} (a \meet/ b_{i}),
  \text{ for all } a \in L \text{ and all families } \{ b_{i} \}_{i \in I} \subseteq L.
$$

\end{enumerate}

\noindent
Define $V \childOf/ U$ (read ``$V$ is included in $U$'') by $a = a \meet/ b$.

\end{Definition}


% ----------------------------------------
\begin{Remark}

The fact that $V \childOf/ U$ is equivalent to $a = a \meet/ b$ means we can deal with the opens of a frame algebraically (via $\meet/$ and $\join/$ operations), or order-theoretically (via the $\childOf/$ relation), whichever is more convenient. 

\end{Remark}

\begin{Remark}

The category of locales is defined as the dual/opposite of the category of frames, and so frames and locales are quite literally the very same objects. In practice, frames are often used for algebraic purposes, and locales are used for (generalized) spatial purposes. Here, we will have no reason to distinguish these two roles, and so we will use the names ``frame'' and ``locale'' interchangeably. 

\end{Remark}

By definition, every locale has a lowest element, which is the join of no regions at all. It represents the absence of any regions whatever. Hence, we typically denote it with the symbol ``$\bottom/$.'' Dually, every locale has a highest element, which is the join of all of the regions. Hence, when convenient we can denote it with the symbol ``$\top$.''


%%%%%%%%%%%%%%%%%%%%%%%%%%%%%%%%%%%%%%%%%%
\subsection{Presentations of locales}

\noindent
Locales have presentations much like groups and other algebraic structures. To give the presentation of a locale, specify a set of generators and relations. 


% ----------------------------------------
\begin{Definition}[Presentation]

A \emph{presentation} $\tuple{G, R}$ of a locale $\category{L}$ is comprised of:

\begin{enumerate}

\item [(P1)] A set of generators $G = \{ U_{k}, U_{m}, \ldots \}$.
\item [(P2)] A set of relations $R \subseteq G \times G$ on those generators.

\end{enumerate}

\noindent
The locale $\category{L}$ presented by $\tuple{G, R}$ is the smallest one freely generated from $G$ which satisfies $R$.

\end{Definition}

% ----------------------------------------
\begin{Remark}

Every locale has a presentation, and a locale can have multiple presentations.

\end{Remark}

To calculate the locale that corresponds to a presentation, start with the generators, then take all finite meets and all arbitrary joins that satisfy $R$ (and of course L1 and L2).


% ----------------------------------------
\begin{Example}
\label{ex:locale-with-overlap-and-bottom}

Let a locale $\category{L}$ be given by the presentation $\tuple{G, R}$ where:

\begin{itemize}

\item $G = \{ \bottom/, U_{1}, U_{2}, U_{3} \}$.
\item $R = \{ \bottom/ \childOf/ U_{1}, U_{1} \childOf/ U_{2}, U_{1} \childOf/ U_{3} \}$.

\end{itemize}

There are four generators ($\bottom/$, $U_{1}$, $U_{2}$, and $U_{3}$), and $\bottom/$ is below $U_{1}$ while $U_{1}$ is a sub-region of $U_{2}$ and $U_{3}$. Since $U_{1}$ is a sub-region of both $U_{2}$ and $U_{3}$, $U_{1}$ is their meet:

\begin{itemize}

\item $U_{1} = U_{2} \meet/ U_{3}$.

\end{itemize}

At this point, we have generated this much of the locale:

\begin{diagram}

\node (U2) at (-2, 1.5) {$U_{2}$};
\node (U3) at (2, 1.5) {$U_{3}$};
\node (U1) at (0, 0) {$U_{1}$};
\node (bottom) at (0, -1) {$\bottom/$};

\draw (bottom) to (U1);
\draw (U1) to (U2);
\draw (U1) to (U3);

\end{diagram}

$R$ says nothing to constrain joins, so we need to join everything we can. In this case, we need to join $U_{2}$ and $U_{3}$:

\begin{diagram}

\node (U2_v_U3) at (0, 3) {$U_{2} \join/ U_{3}$};
\node (U2) at (-2, 1.5) {$U_{2}$};
\node (U3) at (2, 1.5) {$U_{3}$};
\node (U1) at (0, 0) {$U_{1}$};
\node (bottom) at (0, -1) {$\bottom/$};

\draw (bottom) to (U1);
\draw (U1) to (U2);
\draw (U1) to (U3);
\draw (U2) to (U2_v_U3);
\draw (U3) to (U2_v_U3);

\end{diagram}

There are no further joins or meets that aren't already represented in the picture. For instance, all further non-trivial meets are already accounted for:

\begin{itemize}

\item $U_{1} \meet/ \bottom/ = \bottom/$.
\item $U_{2} \meet/ U_{1} = U_{1}$ and $U_{3} \meet/ U_{1} = U_{1}$.
\item $U_{2} \meet/ \bottom/ = \bottom/$ and $U_{3} \meet/ \bottom/ = \bottom/$.
\item $(U_{2} \join/ U_{3}) \meet/ U_{2} = U_{2}$ and $(U_{2} \join/ U_{3}) \meet/ U_{3} = U_{3}$.
\item $(U_{2} \join/ U_{3}) \meet/ U_{1} = U_{1}$.
\item $(U_{2} \join/ U_{3}) \meet/ \bottom/ = \bottom/$.

\end{itemize}

Similarly, all other non-trivial joins are also already accounted for:

\begin{itemize}

\item $\bottom/ \join/ U_{1} = U_{1}$.
\item $\bottom/ \join/ U_{2} = U_{2}$ and $\bottom/ \join/ U_{3} = U_{3}$.
\item $\bottom/ \join/ (U_{2} \join/ U_{3}) = U_{2} \join/ U_{3}$.
\item $U_{1} \join/ U_{2} = U_{2}$ and $U_{1} \join/ U_{3} = U_{3}$.
\item $U_{2} \join/ (U_{2} \join/ U_{3}) = U_{2} \join/ U_{3}$ and $U_{2} \join/ (U_{3} \join/ U_{3}) = U_{2} \join/ U_{3}$.

\end{itemize}

\end{Example}


% ----------------------------------------
\begin{Example}
\label{ex:three-element-free-frame}

Let $\category{L} = \tuple{G, R}$ be given by:

\begin{itemize}

\item $G = \{ U_{1}, U_{2}, U_{3} \}$.
\item $R = \EmptySet/$.

\end{itemize}

We have three generators ($U_{1}$, $U_{2}$, and $U_{3}$), and there are no relations restricting how those generators are related. Thus, the locale that is freely generated from this presentation is isomorphic to the power set of three elements:

\begin{diagram}

\node (123) at (0, 4.5) {$\top = U_{1} \join/ U_{2} \join/ U_{3}$};
\node (12) at (-2, 3) {$U_{1} \join/ U_{2}$};
\node (13) at (0, 3) {$U_{1} \join/ U_{3}$};
\node (23) at (2, 3) {$U_{2} \join/ U_{3}$};
\node (1) at (-2, 1.5) {$U_{1}$};
\node (2) at (0, 1.5) {$U_{2}$};
\node (3) at (2, 1.5) {$U_{3}$};
\node (bottom) at (0, 0) {$\bottom/$};

\draw (bottom) to (1);
\draw (bottom) to (2);
\draw (bottom) to (3);
\draw (1) to (12);
\draw (1) to (13);
\draw (2) to (12);
\draw (2) to (23);
\draw (3) to (13);
\draw (3) to (23);
\draw (12) to (123);
\draw (13) to (123);
\draw (23) to (123);

\end{diagram}

\end{Example}


%%%%%%%%%%%%%%%%%%%%%%%%%%%%%%%%%%%%%%%%%%
\subsection{Presheaves}

\noindent
Above we considered the fibers of a map $f : E \to B$, where $E$ and $B$ were sets. We can also consider fibers over locales, where the fibers respect the locale's structure. This is called a \emph{presheaf}. A presheaf is an assignment of data to each of a locale's regions that is ``stable under restriction,'' i.e., that respects ``zooming'' in and out.

% ----------------------------------------
\begin{Definition}[Presheaf]

Let $\category{L}$ be a locale, and let $\morphisms/(\category{L})$ be $\{ \tuple{A, B} \mid A \childOf/ B \in \category{L} \}$. A \emph{presheaf} on $\category{L}$ is a pair $\tuple{F, \{ \restrict{B}{A} \}_{\tuple{A, B} \in \morphisms/(\category{L})}}$, where:

\begin{itemize}

\item $F$ assigns to each region $U \in L$ some data $F(U)$.

\item $\{ \restrict{B}{A} \}_{\tuple{A, B} \in \morphisms/(\category{L})}$ is a family of maps $\restrict{B}{A} : F(B) \to F(A)$ (called restriction maps), each of which specifies how to restrict the data over $F(B)$ down to the data over $F(A)$.

\end{itemize}

\noindent
All together, $\tuple{F, \{ \restrict{B}{A} \}_{\tuple{A, B} \in \morphisms/(\category{L})}}$ must satisfy the following conditions:

\begin{enumerate}

\item [(R1)] Restrictions preserve identity:
$$\restrict{U}{U} = \ident{U} \text{ (the identity on $U$), for every } U \in \category{L}.$$

\item [(R2)] Restrictions compose:
$$\text{If } A \childOf/ B \text{ and } B \childOf/ C, \text{ then } \restrict{C}{A} = \restrict{B}{A} \compose/ \restrict{C}{B}.$$

\end{enumerate}

\end{Definition}

Since $F$ assigns data $F(U)$ to each region $U \in \category{L}$, we can think of the $F(U)$s as the ``fibers'' over $\category{L}$, and the restriction maps as ``zoom in'' maps that go from bigger fibers down to smaller fibers.

% ----------------------------------------
\begin{Remark}

For the category-theoretically inclined, a presheaf is just a set-valued contravariant functor $F : \oppCategory{L} \to \Set/$. To each region $B$ of $\category{L}$, $F$ assigns to it a set $F(B)$. The contravariance comes from the fact that, to each arrow $B \childOf/ C$ of $\category{L}$, $F$ assigns a restriction map that goes the other way (i.e., that restricts the data from $F(C)$ down to $F(B)$). (R1) and (R2) are automatically satisfied by the fact that $F$ is a functor.

\end{Remark}


% ----------------------------------------
\begin{Example}
\label{ex:presheaf}

Let $\category{L}$ be a locale $\{ \bottom/, W, V, U \}$ with the following structure:

\begin{diagram}

\node (U) at (0, 3) {$U$};
\node (V) at (-2, 1.5) {$V$};
\node (W) at (2, 1.5) {$W$};
\node (bottom) at (0, 0) {$\bottom/$};

\draw (bottom) to (V);
\draw (bottom) to (W);
\draw (V) to (U);
\draw (W) to (U);

\end{diagram}

Define the presheaf $F$ as follows:

\begin{itemize}

\item $F(U) = \{ a, b, c, d \}$, $F(V) = \{ a, b \}$, $F(W) = \{ c, d \}$, $F(\bottom/) = \{ \ast \}$.

\item Define $\restrict{U}{V}$ as the projection (send $a$ to $a$, $b$ to $b$, and the rest can go anywhere), and similarly for $\restrict{U}{W}$. Let $\restrict{U}{\bottom/}$, $\restrict{V}{\bottom/}$, and $\restrict{W}{\bottom/}$ send their data to $\{ \ast \}$, and let the rest be identities.

\end{itemize}

We can see $F$'s assignments as fibers over $\category{L}$ by drawing them over the regions they are assigned to. For instance, over $U$ we have $F(U)$, i.e., $\{ a, b, c, d \}$:

\begin{diagram}

% The locale

\node (U) at (0, 3) {$U$};
\node (V) at (-2, 1.5) {$V$};
\node (W) at (2, 1.5) {$W$};
\node (bottom) at (0, 0) {$\bottom/$};

\draw[dashed] (bottom) to (V);
\draw[dashed] (bottom) to (W);
\draw[dashed] (V) to (U);
\draw[dashed] (W) to (U);

% Fiber over U

\draw (0, 3.25) to (0, 3.775);
\draw (0, 5.25) ellipse (0.5cm and 1.7cm);
\node[dot, label=above:{$a$}] (Ua) at (0, 4) {};
\node[dot, label=above:{$b$}] (Ub) at (0, 4.75) {};
\node[dot, label=above:{$c$}] (Uc) at (0, 5.5) {};
\node[dot, label=above:{$d$}] (Ud) at (0, 6.25) {};

\end{diagram}

Similarly, over $V$ and $W$, we have $F(V) = \{ a, b \}$ and $F(W) = \{ c, d \}$:

\begin{diagram}

% The locale

\node (U) at (0, 3) {$U$};
\node (V) at (-2, 1.5) {$V$};
\node (W) at (2, 1.5) {$W$};
\node (bottom) at (0, 0) {$\bottom/$};

\draw[dashed] (bottom) to (V);
\draw[dashed] (bottom) to (W);
\draw[dashed] (V) to (U);
\draw[dashed] (W) to (U);

% Fiber over U

\draw (0, 3.25) to (0, 3.775);
\draw (0, 5.25) ellipse (0.5cm and 1.7cm);
\node[dot, label=above:{$a$}] (Ua) at (0, 4) {};
\node[dot, label=above:{$b$}] (Ub) at (0, 4.75) {};
\node[dot, label=above:{$c$}] (Uc) at (0, 5.5) {};
\node[dot, label=above:{$d$}] (Ud) at (0, 6.25) {};

% Fiber over V

\draw (-2, 1.75) to (-2, 2.275);
\draw (-2, 3.1) ellipse (0.375cm and 0.875cm);
\node[dot, label=above:{$a$}] (Va) at (-2, 2.5) {};
\node[dot, label=above:{$b$}] (Vb) at (-2, 3.25) {};

% Fiber over W

\draw (2, 1.75) to (2, 2.275);
\draw (2, 3.1) ellipse (0.375cm and 0.875cm);
\node[dot, label=above:{$c$}] (Wc) at (2, 2.5) {};
\node[dot, label=above:{$d$}] (Wd) at (2, 3.25) {};

\end{diagram}

Finally, over $\bottom/$, we have a singleton set:

\begin{diagram}

% The locale

\node (U) at (0, 3) {$U$};
\node (V) at (-2, 1.5) {$V$};
\node (W) at (2, 1.5) {$W$};
\node (bottom) at (0, 0) {$\bottom/$};

\draw[dashed] (bottom) to (V);
\draw[dashed] (bottom) to (W);
\draw[dashed] (V) to (U);
\draw[dashed] (W) to (U);

% Fiber over U

\draw (0, 3.25) to (0, 3.775);
\draw (0, 5.25) ellipse (0.5cm and 1.7cm);
\node[dot, label=above:{$a$}] (Ua) at (0, 4) {};
\node[dot, label=above:{$b$}] (Ub) at (0, 4.75) {};
\node[dot, label=above:{$c$}] (Uc) at (0, 5.5) {};
\node[dot, label=above:{$d$}] (Ud) at (0, 6.25) {};

% Fiber over V

\draw (-2, 1.75) to (-2, 2.275);
\draw (-2, 3.1) ellipse (0.375cm and 0.875cm);
\node[dot, label=above:{$a$}] (Va) at (-2, 2.5) {};
\node[dot, label=above:{$b$}] (Vb) at (-2, 3.25) {};

% Fiber over W

\draw (2, 1.75) to (2, 2.275);
\draw (2, 3.1) ellipse (0.375cm and 0.875cm);
\node[dot, label=above:{$c$}] (Wc) at (2, 2.5) {};
\node[dot, label=above:{$d$}] (Wd) at (2, 3.25) {};

% Fiber over bottom

\draw (0, 0.25) to (0, 0.75);
\draw (0, 1.25) ellipse (0.375cm and 0.575cm);
\node[dot, label=above:{$\ast$}] (bottom) at (0, 1) {};

\end{diagram}

The restriction maps show how to ``zoom in'' on the data over each region. For instance, $\restrict{U}{V}$ shows how to restrict the data in the fiber over $U$ down to the data in the fiber over $V$:

\begin{diagram}

% The locale

\node (U) at (0, 3) {$U$};
\node (V) at (-2, 1.5) {$V$};
\node (W) at (2, 1.5) {$W$};
\node (bottom) at (0, 0) {$\bottom/$};

\draw[dashed] (bottom) to (V);
\draw[dashed] (bottom) to (W);
\draw[dashed] (V) to (U);
\draw[dashed] (W) to (U);

% Fiber over U

\draw (0, 3.25) to (0, 3.775);
\draw (0, 5.25) ellipse (0.5cm and 1.7cm);
\node[dot, label=above:{$a$}] (Ua) at (0, 4) {};
\node[dot, label=above:{$b$}] (Ub) at (0, 4.75) {};
\node[dot, label=above:{$c$}] (Uc) at (0, 5.5) {};
\node[dot, label=above:{$d$}] (Ud) at (0, 6.25) {};

% Fiber over V

\draw (-2, 1.75) to (-2, 2.275);
\draw (-2, 3.1) ellipse (0.375cm and 0.875cm);
\node[dot, label=above:{$a$}] (Va) at (-2, 2.5) {};
\node[dot, label=above:{$b$}] (Vb) at (-2, 3.25) {};

% Fiber over W

\draw (2, 1.75) to (2, 2.275);
\draw (2, 3.1) ellipse (0.375cm and 0.875cm);
\node[dot, label=above:{$c$}] (Wc) at (2, 2.5) {};
\node[dot, label=above:{$d$}] (Wd) at (2, 3.25) {};

% Fiber over bottom

\draw (0, 0.25) to (0, 0.75);
\draw (0, 1.25) ellipse (0.375cm and 0.575cm);
\node[dot, label=above:{$\ast$}] (bottom) at (0, 1) {};

% rho U, V

\draw[arrow, ->] (Ud.west) to (Vb.east);
\draw[arrow, ->] (Uc.west) to (Vb.east);
\draw[arrow, ->] (Ub.west) to (Vb.east);
\draw[arrow, ->] (Ua.west) to (Va.east);
\node at (-1.675, 4.5) {$\restrict{U}{V}$};

\end{diagram}

It's similar for the fiber over $W$:

\begin{diagram}

% The locale

\node (U) at (0, 3) {$U$};
\node (V) at (-2, 1.5) {$V$};
\node (W) at (2, 1.5) {$W$};
\node (bottom) at (0, 0) {$\bottom/$};

\draw[dashed] (bottom) to (V);
\draw[dashed] (bottom) to (W);
\draw[dashed] (V) to (U);
\draw[dashed] (W) to (U);

% Fiber over U

\draw (0, 3.25) to (0, 3.775);
\draw (0, 5.25) ellipse (0.5cm and 1.7cm);
\node[dot, label=above:{$a$}] (Ua) at (0, 4) {};
\node[dot, label=above:{$b$}] (Ub) at (0, 4.75) {};
\node[dot, label=above:{$c$}] (Uc) at (0, 5.5) {};
\node[dot, label=above:{$d$}] (Ud) at (0, 6.25) {};

% Fiber over V

\draw (-2, 1.75) to (-2, 2.275);
\draw (-2, 3.1) ellipse (0.375cm and 0.875cm);
\node[dot, label=above:{$a$}] (Va) at (-2, 2.5) {};
\node[dot, label=above:{$b$}] (Vb) at (-2, 3.25) {};

% Fiber over W

\draw (2, 1.75) to (2, 2.275);
\draw (2, 3.1) ellipse (0.375cm and 0.875cm);
\node[dot, label=above:{$c$}] (Wc) at (2, 2.5) {};
\node[dot, label=above:{$d$}] (Wd) at (2, 3.25) {};

% Fiber over bottom

\draw (0, 0.25) to (0, 0.75);
\draw (0, 1.25) ellipse (0.375cm and 0.575cm);
\node[dot, label=above:{$\ast$}] (bottom) at (0, 1) {};

% rho U, W

\draw[arrow, ->] (Ud.east) to (Wd.west);
\draw[arrow, ->] (Uc.east) to (Wc.west);
\draw[arrow, ->] (Ub.east) to (Wc.west);
\draw[arrow, ->] (Ua.east) to (Wc.west);
\node at (1.675, 4.5) {$\restrict{U}{W}$};

\end{diagram}

Restricting a fiber to itself is just the identity on the fiber:

\begin{diagram}

% The locale

\node (U) at (0, 3) {$U$};
\node (V) at (-2, 1.5) {$V$};
\node (W) at (2, 1.5) {$W$};
\node (bottom) at (0, 0) {$\bottom/$};

\draw[dashed] (bottom) to (V);
\draw[dashed] (bottom) to (W);
\draw[dashed] (V) to (U);
\draw[dashed] (W) to (U);

% Fiber over U

\draw (0, 3.25) to (0, 3.775);
\draw (0, 5.25) ellipse (0.5cm and 1.7cm);
\node[dot, label=above:{$a$}] (Ua) at (0, 4) {};
\node[dot, label=above:{$b$}] (Ub) at (0, 4.75) {};
\node[dot, label=above:{$c$}] (Uc) at (0, 5.5) {};
\node[dot, label=above:{$d$}] (Ud) at (0, 6.25) {};

% Fiber over V

\draw (-2, 1.75) to (-2, 2.275);
\draw (-2, 3.1) ellipse (0.375cm and 0.875cm);
\node[dot, label=above:{$a$}] (Va) at (-2, 2.5) {};
\node[dot, label=above:{$b$}] (Vb) at (-2, 3.25) {};

% Fiber over W

\draw (2, 1.75) to (2, 2.275);
\draw (2, 3.1) ellipse (0.375cm and 0.875cm);
\node[dot, label=above:{$c$}] (Wc) at (2, 2.5) {};
\node[dot, label=above:{$d$}] (Wd) at (2, 3.25) {};

% Fiber over bottom

\draw (0, 0.25) to (0, 0.75);
\draw (0, 1.25) ellipse (0.375cm and 0.575cm);
\node[dot, label=above:{$\ast$}] (bottom) at (0, 1) {};

% rho V, V

\draw[arrow, ->] (Vb) to[out=215,in=155,looseness=35] (Vb);
\draw[arrow, ->] (Va) to[out=215,in=155,looseness=35] (Va);
\node at (-3.175, 2.875) {$\restrict{V}{V}$};

\end{diagram}

The other restriction maps restrict down to the singleton set. For instance:

\begin{diagram}

% The locale

\node (U) at (0, 3) {$U$};
\node (V) at (-2, 1.5) {$V$};
\node (W) at (2, 1.5) {$W$};
\node (bottom) at (0, 0) {$\bottom/$};

\draw[dashed] (bottom) to (V);
\draw[dashed] (bottom) to (W);
\draw[dashed] (V) to (U);
\draw[dashed] (W) to (U);

% Fiber over U

\draw (0, 3.25) to (0, 3.775);
\draw (0, 5.25) ellipse (0.5cm and 1.7cm);
\node[dot, label=above:{$a$}] (Ua) at (0, 4) {};
\node[dot, label=above:{$b$}] (Ub) at (0, 4.75) {};
\node[dot, label=above:{$c$}] (Uc) at (0, 5.5) {};
\node[dot, label=above:{$d$}] (Ud) at (0, 6.25) {};

% Fiber over V

\draw (-2, 1.75) to (-2, 2.275);
\draw (-2, 3.1) ellipse (0.375cm and 0.875cm);
\node[dot, label=above:{$a$}] (Va) at (-2, 2.5) {};
\node[dot, label=above:{$b$}] (Vb) at (-2, 3.25) {};

% Fiber over W

\draw (2, 1.75) to (2, 2.275);
\draw (2, 3.1) ellipse (0.375cm and 0.875cm);
\node[dot, label=above:{$c$}] (Wc) at (2, 2.5) {};
\node[dot, label=above:{$d$}] (Wd) at (2, 3.25) {};

% Fiber over bottom

\draw (0, 0.25) to (0, 0.75);
\draw (0, 1.25) ellipse (0.375cm and 0.575cm);
\node[dot, label=above:{$\ast$}] (bottom) at (0, 1) {};

% rho V, bottom

\draw[arrow, ->] (Vb.east) to (bottom.west);
\draw[arrow, ->] (Va.east) to (bottom.west);
\node at (-0.45, 2.15) {$\restrict{V}{\bottom/}$};

\end{diagram}

All of this makes it clear that the structure of the presheaf data that sits in the fibers over $\category{L}$ mimics (respects) the structure of the base locale:

\begin{diagram}

% The locale

\node (U) at (0, 3) {$U$};
\node (V) at (-2, 1.5) {$V$};
\node (W) at (2, 1.5) {$W$};
\node (bottom) at (0, 0) {$\bottom/$};

\draw[dashed] (bottom) to (V);
\draw[dashed] (bottom) to (W);
\draw[dashed] (V) to (U);
\draw[dashed] (W) to (U);

% Fiber over U

\draw (0, 3.25) to (0, 3.775);
\draw (0, 5.25) ellipse (0.5cm and 1.7cm);
\node[dot, label=above:{$a$}] (Ua) at (0, 4) {};
\node[dot, label=above:{$b$}] (Ub) at (0, 4.75) {};
\node[dot, label=above:{$c$}] (Uc) at (0, 5.5) {};
\node[dot, label=above:{$d$}] (Ud) at (0, 6.25) {};

% Fiber over V

\draw (-2, 1.75) to (-2, 2.275);
\draw (-2, 3.1) ellipse (0.375cm and 0.875cm);
\node[dot, label=above:{$a$}] (Va) at (-2, 2.5) {};
\node[dot, label=above:{$b$}] (Vb) at (-2, 3.25) {};

% Fiber over W

\draw (2, 1.75) to (2, 2.275);
\draw (2, 3.1) ellipse (0.375cm and 0.875cm);
\node[dot, label=above:{$c$}] (Wc) at (2, 2.5) {};
\node[dot, label=above:{$d$}] (Wd) at (2, 3.25) {};

% Fiber over bottom

\draw (0, 0.25) to (0, 0.75);
\draw (0, 1.25) ellipse (0.375cm and 0.575cm);
\node[dot, label=above:{$\ast$}] (bottom) at (0, 1) {};

% rho U, V

\draw[arrow, ->] (Ud.west) to (Vb.east);
\draw[arrow, ->] (Uc.west) to (Vb.east);
\draw[arrow, ->] (Ub.west) to (Vb.east);
\draw[arrow, ->] (Ua.west) to (Va.east);
\node at (-1.675, 4.5) {$\restrict{U}{V}$};

% rho U, W

\draw[arrow, ->] (Ud.east) to (Wd.west);
\draw[arrow, ->] (Uc.east) to (Wc.west);
\draw[arrow, ->] (Ub.east) to (Wc.west);
\draw[arrow, ->] (Ua.east) to (Wc.west);
\node at (1.675, 4.5) {$\restrict{U}{W}$};

% rho V, bottom

\draw[arrow, ->] (Vb.east) to (bottom.west);
\draw[arrow, ->] (Va.east) to (bottom.west);
\node at (-0.45, 2.15) {$\restrict{V}{\bottom/}$};

% rho W, bottom

\draw[arrow, ->] (Wd.west) to (bottom.east);
\draw[arrow, ->] (Wc.west) to (bottom.east);
\node at (0.45, 2.15) {$\restrict{W}{\bottom/}$};

\end{diagram}

\end{Example}


%%%%%%%%%%%%%%%%%%%%%%%%%%%%%%%%%%%%%%%%%%
\subsection{Gluing}

\noindent
The definition of a presheaf requires only that the data be stable under restriction (zooming in on a region). It does not require that the data fit together across different regions (fibers).

In some cases though, certain sections in different fibers turn out to be compatible (i.e., there is a coherent way to patch them together). When this occurs, those compatible sections can be glued together to form sections that stretch across fibers.

To get at this idea, let's first define a \emph{cover}. A cover of a region $U$ is a collection of sub-regions that, when combined, contain $U$ in its entirety. The chosen sub-regions don't leave any part of $U$ exposed. For example, the Northern and Southern hemispheres are a cover of the Earth. So are the Southern, Eastern, and Western hemispheres.  The Northern and Eastern hemispheres do \emph{not} form a cover of the Earth, because when taken together, they lack portions of the Southern and Western hemispheres.

% ----------------------------------------
\begin{Definition}[Cover]

Let $\category{L}$ be a locale, and let $U$ be a region of $\category{L}$. A \emph{cover} of $U$ is a family $\{ U_{i} \}_{i \in I} \subseteq \category{L}$ such that:

$$U = \bigjoin/\limits_{i \in I} \{ U_{i} \}.$$

\noindent
In other words, a cover of $U$ is a family of regions that join together to form $U$.

\end{Definition}

\begin{Remark}

Recall that when we view a topology as a locale, ``join'' denotes the union of open sets.  Thus, our definition of cover coincides with the traditional sense of \emph{open cover} familiar to topologists.

\end{Remark}

% ----------------------------------------
\begin{Example}

Take the topology from \cref{ex:topology}: $T = \{ \EmptySet/$, $\{ b \}$, $\{ a, b \}$, $\{ b, c \}$, $\{a, b, c \} \}$. A cover of $\{ a, b, c \}$ is $\{ a, b \}$ and $\{ b, c \}$, because altogether, $\{ a, b \}$ and $\{ b, c \}$ cover all of the points in $\{ a, b, c \}$.

Another cover of $\{ a, b, c \}$ is $\{ \{ a, b \}, \{ b, c \}, \{ b \} \}$. Although $\{ b \}$ is redundant here, this choice of sub-regions still entirely covers $\{ a, b, c \}$ as required.

\end{Example}


% ----------------------------------------
\begin{Example}

In the context of locales, where there are no points, a cover of $U$ is just a selection of sub-regions of $U$ that join together to form $U$. Take the locale from \cref{ex:presheaf}:

\begin{diagram}

\node (U) at (0, 3) {$U$};
\node (V) at (-2, 1.5) {$V$};
\node (W) at (2, 1.5) {$W$};
\node (bottom) at (0, 0) {$\bottom/$};

\draw (bottom) to (V);
\draw (bottom) to (W);
\draw (V) to (U);
\draw (W) to (U);

\end{diagram}

\noindent
A cover of $U$ is $\{ V, W \}$, since $U = \bigjoin/ \{ V, W \}$:

\begin{diagram}

\draw[fill=selected] (-2, 1.5) ellipse (0.5cm and 0.5cm);
\draw[fill=selected] (2, 1.5) ellipse (0.5cm and 0.5cm);

\node (U) at (0, 3) {$U$};
\node (V) at (-2, 1.5) {$V$};
\node (W) at (2, 1.5) {$W$};
\node (bottom) at (0, 0) {\textcolor{faded}{$\bottom/$}};

\draw[faded] (bottom) to (V);
\draw[faded] (bottom) to (W);
\draw (V) to (U);
\draw (W) to (U);

\end{diagram}

A cover of $V$ is just $\{ V \}$:

\begin{diagram}

\draw[fill=selected] (-2, 1.5) ellipse (0.5cm and 0.5cm);

\node (U) at (0, 3) {\textcolor{faded}{$U$}};
\node (V) at (-2, 1.5) {$V$};
\node (W) at (2, 1.5) {\textcolor{faded}{$W$}};
\node (bottom) at (0, 0) {\textcolor{faded}{$\bottom/$}};

\draw[faded] (bottom) to (V);
\draw[faded] (bottom) to (W);
\draw[faded] (V) to (U);
\draw[faded] (W) to (U);

\end{diagram}
 
\end{Example}


% ----------------------------------------
\begin{Remark}

A cover over the least element of a locale (or a topology) is empty (the empty set), because there are no regions (or points) to cover. 

\end{Remark}


Given a presheaf $F$ over a locale $\category{L}$, if we have a cover $\{ U_{i} \}_{i \in I}$ of some portion of $\category{L}$, there is a corresponding family of fibers $\{ F(U_{i}) \}_{i \in I}$ over that cover.  We can pick one section (i.e., one element) from each such fiber to get a slice of elements that spans all of the fibers over that cover. Let us call such a choice a \emph{selection of patch candidates}.


% ----------------------------------------
\begin{Definition}[Patch candidates]

Given a presheaf $F$ and a cover $\{ U_{i} \}_{i \in I}$ with a corresponding family of fibers $\{ F(U_{i}) \}_{i \in I}$, a \emph{selection of patch candidates} $\{ s_{i} \}_{i \in I}$ is a choice of one section $s_{i}$ from each $F(U_{i})$.

\end{Definition}


% ----------------------------------------
\begin{Example}

Take the presheaf from \cref{ex:presheaf}:

\begin{diagram}

\node (U) at (0, 3) {$U$};
\node (V) at (-2, 1.5) {$V$};
\node (W) at (2, 1.5) {$W$};
\node (bottom) at (0, 0) {$\bottom/$};

\draw[dashed] (bottom) to (V);
\draw[dashed] (bottom) to (W);
\draw[dashed] (V) to (U);
\draw[dashed] (W) to (U);

\draw (0, 3.25) to (0, 3.775);
\draw (0, 5.25) ellipse (0.5cm and 1.7cm);
\node[dot, label=above:{$a$}] (Ua) at (0, 4) {};
\node[dot, label=above:{$b$}] (Ub) at (0, 4.75) {};
\node[dot, label=above:{$c$}] (Uc) at (0, 5.5) {};
\node[dot, label=above:{$d$}] (Ud) at (0, 6.25) {};

\draw (-2, 1.75) to (-2, 2.275);
\draw (-2, 3.1) ellipse (0.375cm and 0.875cm);
\node[dot, label=above:{$a$}] at (-2, 2.5) {};
\node[dot, label=above:{$b$}] at (-2, 3.25) {};

\draw (2, 1.75) to (2, 2.275);
\draw (2, 3.1) ellipse (0.375cm and 0.875cm);
\node[dot, label=above:{$c$}] at (2, 2.5) {};
\node[dot, label=above:{$d$}] at (2, 3.25) {};

\draw (0, 0.25) to (0, 0.75);
\draw (0, 1.25) ellipse (0.375cm and 0.575cm);
\node[dot, label=above:{$\ast$}] at (0, 1) {};

\end{diagram}

Let $\{ V, W \}$ be the cover of interest:

\begin{diagram}

\draw[fill=selected] (-2, 1.5) ellipse (0.5cm and 0.5cm);
\draw[fill=selected] (2, 1.5) ellipse (0.5cm and 0.5cm);

\node (U) at (0, 3) {\textcolor{faded}{$U$}};
\node (V) at (-2, 1.5) {$V$};
\node (W) at (2, 1.5) {$W$};
\node (bottom) at (0, 0) {\textcolor{faded}{$\bottom/$}};

\draw[dashed,faded] (bottom) to (V);
\draw[dashed,faded] (bottom) to (W);
\draw[dashed,faded] (V) to (U);
\draw[dashed,faded] (W) to (U);

\draw[faded] (0, 3.25) to (0, 3.775);
\draw[faded] (0, 5.25) ellipse (0.5cm and 1.7cm);
\node[dot, label=above:{\textcolor{faded}{$a$}}, faded] at (0, 4) {};
\node[dot, label=above:{\textcolor{faded}{$b$}}, faded] at (0, 4.75) {};
\node[dot, label=above:{\textcolor{faded}{$c$}}, faded] at (0, 5.5) {};
\node[dot, label=above:{\textcolor{faded}{$d$}}, faded] at (0, 6.25) {};

\draw[faded] (-2, 1.75) to (-2, 2.275);
\draw[faded] (-2, 3.1) ellipse (0.375cm and 0.875cm);
\node[dot, label=above:{\textcolor{faded}{$a$}}, faded] at (-2, 2.5) {};
\node[dot, label=above:{\textcolor{faded}{$b$}}, faded] at (-2, 3.25) {};

\draw[faded] (2, 1.75) to (2, 2.275);
\draw[faded] (2, 3.1) ellipse (0.375cm and 0.875cm);
\node[dot, label=above:{\textcolor{faded}{$c$}}, faded] at (2, 2.5) {};
\node[dot, label=above:{\textcolor{faded}{$d$}}, faded] at (2, 3.25) {};

\draw[faded] (0, 0.25) to (0, 0.75);
\draw[faded] (0, 1.25) ellipse (0.375cm and 0.575cm);
\node[dot, label=above:{\textcolor{faded}{$\ast$}}, faded] at (0, 1) {};

\end{diagram}

Over this cover, we have a corresponding family of fibers:

\begin{diagram}

\node (U) at (0, 3) {\textcolor{faded}{$U$}};
\node (V) at (-2, 1.5) {$V$};
\node (W) at (2, 1.5) {$W$};
\node (bottom) at (0, 0) {\textcolor{faded}{$\bottom/$}};

\draw[dashed,faded] (bottom) to (V);
\draw[dashed,faded] (bottom) to (W);
\draw[dashed,faded] (V) to (U);
\draw[dashed,faded] (W) to (U);

\draw[faded] (0, 3.25) to (0, 3.775);
\draw[faded] (0, 5.25) ellipse (0.5cm and 1.7cm);
\node[dot, label=above:{\textcolor{faded}{$a$}}, faded] at (0, 4) {};
\node[dot, label=above:{\textcolor{faded}{$b$}}, faded] at (0, 4.75) {};
\node[dot, label=above:{\textcolor{faded}{$c$}}, faded] at (0, 5.5) {};
\node[dot, label=above:{\textcolor{faded}{$d$}}, faded] at (0, 6.25) {};

\draw (-2, 1.75) to (-2, 2.275);
\draw (-2, 3.1) ellipse (0.375cm and 0.875cm);
\node[dot, label=above:{$a$}] at (-2, 2.5) {};
\node[dot, label=above:{$b$}] at (-2, 3.25) {};

\draw (2, 1.75) to (2, 2.275);
\draw (2, 3.1) ellipse (0.375cm and 0.875cm);
\node[dot, label=above:{$c$}] at (2, 2.5) {};
\node[dot, label=above:{$d$}] at (2, 3.25) {};

\draw[faded] (0, 0.25) to (0, 0.75);
\draw[faded] (0, 1.25) ellipse (0.375cm and 0.575cm);
\node[dot, label=above:{\textcolor{faded}{$\ast$}}, faded] at (0, 1) {};

\end{diagram}

A selection of patch candidates is a choice of one section (element) from each fiber. For instance, we might pick $b$ from $F(V)$ and $c$ from $F(W)$:

\begin{diagram}

\draw[rounded corners=4pt,fill=selected] (-2.5, 3.4) rectangle (-1.5, 3.1);
\draw[rounded corners=4pt,fill=selected] (1.5, 2.65) rectangle (2.5, 2.35);

\node (U) at (0, 3) {\textcolor{faded}{$U$}};
\node (V) at (-2, 1.5) {$V$};
\node (W) at (2, 1.5) {$W$};
\node (bottom) at (0, 0) {\textcolor{faded}{$\bottom/$}};

\draw[dashed,faded] (bottom) to (V);
\draw[dashed,faded] (bottom) to (W);
\draw[dashed,faded] (V) to (U);
\draw[dashed,faded] (W) to (U);

\draw[faded] (0, 3.25) to (0, 3.775);
\draw[faded] (0, 5.25) ellipse (0.5cm and 1.7cm);
\node[dot, label=above:{\textcolor{faded}{$a$}}, faded] at (0, 4) {};
\node[dot, label=above:{\textcolor{faded}{$b$}}, faded] at (0, 4.75) {};
\node[dot, label=above:{\textcolor{faded}{$c$}}, faded] at (0, 5.5) {};
\node[dot, label=above:{\textcolor{faded}{$d$}}, faded] at (0, 6.25) {};

\draw (-2, 1.75) to (-2, 2.275);
\draw (-2, 3.1) ellipse (0.375cm and 0.875cm);
\node[dot, label=above:{$a$}] at (-2, 2.5) {};
\node[dot, label=above:{$b$}] at (-2, 3.25) {};

\draw (2, 1.75) to (2, 2.275);
\draw (2, 3.1) ellipse (0.375cm and 0.875cm);
\node[dot, label=above:{$c$}] at (2, 2.5) {};
\node[dot, label=above:{$d$}] at (2, 3.25) {};

\draw[faded] (0, 0.25) to (0, 0.75);
\draw[faded] (0, 1.25) ellipse (0.375cm and 0.575cm);
\node[dot, label=above:{\textcolor{faded}{$\ast$}}, faded] at (0, 1) {};

\end{diagram}

Similarly, we might pick $\{ a, d \}$, $\{ b, d \}$, or $\{a, c \}$, each of which is a valid selection of patch candidates.

\end{Example}

% ----------------------------------------
\begin{Example}

Consider the empty cover. Since there are no sub-regions below the least element of a locale, there are no patch candidates we could choose for the empty cover either. Hence, any selection of patch candidates for the empty cover is $\EmptySet/$.

\end{Example}


A selection of patch candidates might fit together, or they might not. We say they are compatible if they fit together, i.e., if they agree on overlaps. To check this, take any pair of patch candidates, and check if they restrict to the same data on their overlap.

% ----------------------------------------
\begin{Definition}[Compatible patch candidates]

Given two fibers $F(U_{i})$ and $F(U_{j})$ and a patch candidate from each, $s_{i} \in F(U_{i})$ and $s_{j} \in F(U_{j})$, $s_{i}$ and $s_{j}$ are \emph{compatible} if they restrict to the same data on their overlap $U_{i} \meet/ U_{j}$:
$$\restrict{U_{i}}{U_{i} \meet/ U_{j}}(s_{i}) 
    = 
    \restrict{U_{j}}{U_{i} \meet/ U_{j}}(s_{j}).$$

\noindent
A selection of patch candidates $\{ s_{i} \}_{i \in I}$ is compatible if all of its members are pair-wise compatible.

\end{Definition}


% ----------------------------------------
\begin{Example}
\label{ex:compatible-patch-candidates}

Consider the following presheaf $F$:

\begin{diagram}

\node (U) at (0, 3) {$U$};
\node (V) at (-2, 1.5) {$V$};
\node (W) at (2, 1.5) {$W$};
\node (bottom) at (0, 0) {$\bottom/$};

\draw[dashed] (bottom) to (V);
\draw[dashed] (bottom) to (W);
\draw[dashed] (V) to (U);
\draw[dashed] (W) to (U);

\draw (0, 3.25) to (0, 3.75);
\draw (0, 4.925) ellipse (0.425cm and 1.325cm);
\node[dot, label=above:{$a$}] (Ua) at (0, 4) {};
\node[dot, label=above:{$b$}] (Ub) at (0, 4.75) {};
\node[dot, label=above:{$c$}] (Uc) at (0, 5.5) {};

\draw (-2, 1.75) to (-2, 2.275);
\draw (-2, 3.1) ellipse (0.375cm and 0.875cm);
\node[dot, label=above:{$s$}] (Vs) at (-2, 2.5) {};
\node[dot, label=above:{$r$}] (Vr) at (-2, 3.25) {};

\draw (2, 1.75) to (2, 2.275);
\draw (2, 3.1) ellipse (0.375cm and 0.875cm);
\node[dot, label=above:{$u$}] (Wu) at (2, 2.5) {};
\node[dot, label=above:{$t$}] (Wt) at (2, 3.25) {};

\draw (0, 0.25) to (0, 0.75);
\draw (0, 1.5) ellipse (0.375cm and 0.875cm);
\node[dot, label=above:{$p$}] (bottomp) at (0, 1) {};
\node[dot, label=above:{$q$}] (bottomq) at (0, 1.75) {};

\draw[arrow, ->] (Uc) to (Vr);
\draw[arrow, ->] (Ub) to (Vr);
\draw[arrow, ->] (Ua) to (Vs);
\node at (-1.25, 4.5) {$\restrict{U}{V}$};

\draw[arrow, ->] (Ua) to (Wu);
\draw[arrow, ->] (Ub) to (Wu);
\draw[arrow, ->] (Uc) to (Wt);
\node at (1.675, 4.5) {$\restrict{U}{W}$};

\draw[arrow, ->] (Vr) to (bottomq);
\draw[arrow, ->] (Vs) to (bottomp);
\node at (-1.25, 1.5) {$\restrict{V}{\bottom/}$};

\draw[arrow, ->] (Wu) to (bottomq);
\draw[arrow, ->] (Wt) to (bottomp);
\node at (1, 1.5) {$\restrict{W}{\bottom/}$};

\end{diagram}

Take the cover $\{ V, W \}$ and its corresponding fibers:

\begin{diagram}

\node (U) at (0, 3) {\textcolor{faded}{$U$}};
\node (V) at (-2, 1.5) {$V$};
\node (W) at (2, 1.5) {$W$};
\node (bottom) at (0, 0) {\textcolor{faded}{$\bottom/$}};

\draw[dashed,faded] (bottom) to (V);
\draw[dashed,faded] (bottom) to (W);
\draw[dashed,faded] (V) to (U);
\draw[dashed,faded] (W) to (U);

\draw (-2, 1.75) to (-2, 2.275);
\draw (-2, 3.1) ellipse (0.375cm and 0.875cm);
\node[dot, label=above:{$s$}] (Vs) at (-2, 2.5) {};
\node[dot, label=above:{$r$}] (Vr) at (-2, 3.25) {};

\draw (2, 1.75) to (2, 2.275);
\draw (2, 3.1) ellipse (0.375cm and 0.875cm);
\node[dot, label=above:{$u$}] (Wu) at (2, 2.5) {};
\node[dot, label=above:{$t$}] (Wt) at (2, 3.25) {};

\end{diagram}

Suppose we pick $\{ s, t \}$ for patch candidates:

\begin{diagram}

\draw[rounded corners=4pt,fill=selected] (-2.5, 2.65) rectangle (-1.5, 2.35);
\draw[rounded corners=4pt,fill=selected] (1.5, 3.4) rectangle (2.5, 3.1);

\node (U) at (0, 3) {\textcolor{faded}{$U$}};
\node (V) at (-2, 1.5) {$V$};
\node (W) at (2, 1.5) {$W$};
\node (bottom) at (0, 0) {\textcolor{faded}{$\bottom/$}};

\draw[dashed,faded] (bottom) to (V);
\draw[dashed,faded] (bottom) to (W);
\draw[dashed,faded] (V) to (U);
\draw[dashed,faded] (W) to (U);

\draw (-2, 1.75) to (-2, 2.275);
\draw (-2, 3.1) ellipse (0.375cm and 0.875cm);
\node[dot, label=above:{$s$}] (Vs) at (-2, 2.5) {};
\node[dot, label=above:{$r$}] (Vr) at (-2, 3.25) {};

\draw (2, 1.75) to (2, 2.275);
\draw (2, 3.1) ellipse (0.375cm and 0.875cm);
\node[dot, label=above:{$u$}] (Wu) at (2, 2.5) {};
\node[dot, label=above:{$t$}] (Wt) at (2, 3.25) {};

\end{diagram}

Is this selection compatible? We have to check if they agree on their overlap. The overlap $V \meet/ W$ is $\bottom/$. Where does $\restrict{V}{\bottom/}$ send our chosen patch candidate $s$? It sends it to $p$, since $\restrict{V}{\bottom/}(s) = p$. Where does $\restrict{W}{\bottom/}$ send our other chosen patch candidate $t$? It also sends it to $p$, since $\restrict{W}{\bottom/}(t) = p$. On the overlap $\bottom/$ then, $\restrict{V}{\bottom/}(s) = \restrict{W}{\bottom/}(t)$, so $s$ and $t$ are compatible. This is easy to see in the diagram, since $s$ and $t$ both get sent to the same place:

\begin{diagram}

\draw[rounded corners=4pt,fill=selected] (-2.5, 2.65) rectangle (-1.5, 2.35);
\draw[rounded corners=4pt,fill=selected] (1.5, 3.4) rectangle (2.5, 3.1);

\node (U) at (0, 3) {\textcolor{faded}{$U$}};
\node (V) at (-2, 1.5) {$V$};
\node (W) at (2, 1.5) {$W$};
\node (bottom) at (0, 0) {$\bottom/$};

\draw[dashed,faded] (bottom) to (V);
\draw[dashed,faded] (bottom) to (W);
\draw[dashed,faded] (V) to (U);
\draw[dashed,faded] (W) to (U);

\draw (-2, 1.75) to (-2, 2.275);
\draw (-2, 3.1) ellipse (0.375cm and 0.875cm);
\node[dot, label=above:{$s$}] (Vs) at (-2, 2.5) {};
\node[dot, label=above:{$r$}] (Vr) at (-2, 3.25) {};

\draw (2, 1.75) to (2, 2.275);
\draw (2, 3.1) ellipse (0.375cm and 0.875cm);
\node[dot, label=above:{$u$}] (Wu) at (2, 2.5) {};
\node[dot, label=above:{$t$}] (Wt) at (2, 3.25) {};

\draw (0, 0.25) to (0, 0.75);
\draw (0, 1.5) ellipse (0.375cm and 0.875cm);
\node[dot, label=above:{$p$}] (bottomp) at (0, 1) {};
\node[dot, label=above:{\textcolor{faded}{$q$}}, faded] (bottomq) at (0, 1.75) {};

\draw[arrow, ->, faded] (Vr) to (bottomq);
\draw[arrow, ->] (Vs) to (bottomp);
\node at (-1.25, 1.5) {$\restrict{V}{\bottom/}$};

\draw[arrow, ->, faded] (Wu) to (bottomq);
\draw[arrow, ->] (Wt) to (bottomp);
\node at (1, 1.5) {$\restrict{W}{\bottom/}$};

\end{diagram}

Now suppose we pick $\{ r, u \}$ for patch candidates:

\begin{diagram}

\draw[rounded corners=4pt,fill=selected] (-2.5, 3.4) rectangle (-1.5, 3.1);
\draw[rounded corners=4pt,fill=selected] (1.5, 2.65) rectangle (2.5, 2.35);

\node (U) at (0, 3) {\textcolor{faded}{$U$}};
\node (V) at (-2, 1.5) {$V$};
\node (W) at (2, 1.5) {$W$};
\node (bottom) at (0, 0) {\textcolor{faded}{$\bottom/$}};

\draw[dashed,faded] (bottom) to (V);
\draw[dashed,faded] (bottom) to (W);
\draw[dashed,faded] (V) to (U);
\draw[dashed,faded] (W) to (U);

\draw (-2, 1.75) to (-2, 2.275);
\draw (-2, 3.1) ellipse (0.375cm and 0.875cm);
\node[dot, label=above:{$s$}] (Vs) at (-2, 2.5) {};
\node[dot, label=above:{$r$}] (Vr) at (-2, 3.25) {};

\draw (2, 1.75) to (2, 2.275);
\draw (2, 3.1) ellipse (0.375cm and 0.875cm);
\node[dot, label=above:{$u$}] (Wu) at (2, 2.5) {};
\node[dot, label=above:{$t$}] (Wt) at (2, 3.25) {};

\end{diagram}

These are also compatible. They agree on their overlap (both restrict to $q$):

\begin{diagram}

\draw[rounded corners=4pt,fill=selected] (-2.5, 3.4) rectangle (-1.5, 3.1);
\draw[rounded corners=4pt,fill=selected] (1.5, 2.65) rectangle (2.5, 2.35);

\node (U) at (0, 3) {\textcolor{faded}{$U$}};
\node (V) at (-2, 1.5) {$V$};
\node (W) at (2, 1.5) {$W$};
\node (bottom) at (0, 0) {$\bottom/$};

\draw[dashed,faded] (bottom) to (V);
\draw[dashed,faded] (bottom) to (W);
\draw[dashed,faded] (V) to (U);
\draw[dashed,faded] (W) to (U);

\draw (-2, 1.75) to (-2, 2.275);
\draw (-2, 3.1) ellipse (0.375cm and 0.875cm);
\node[dot, label=above:{$s$}] (Vs) at (-2, 2.5) {};
\node[dot, label=above:{$r$}] (Vr) at (-2, 3.25) {};

\draw (2, 1.75) to (2, 2.275);
\draw (2, 3.1) ellipse (0.375cm and 0.875cm);
\node[dot, label=above:{$u$}] (Wu) at (2, 2.5) {};
\node[dot, label=above:{$t$}] (Wt) at (2, 3.25) {};

\draw (0, 0.25) to (0, 0.75);
\draw (0, 1.5) ellipse (0.375cm and 0.875cm);
\node[dot, label=above:{\textcolor{faded}{$p$}}, faded] (bottomp) at (0, 1) {};
\node[dot, label=above:{$q$}] (bottomq) at (0, 1.75) {};

\draw[arrow, ->] (Vr) to (bottomq);
\draw[arrow, ->, faded] (Vs) to (bottomp);
\node at (-1.25, 1.5) {$\restrict{V}{\bottom/}$};

\draw[arrow, ->, faded] (Wt) to (bottomp);
\draw[arrow, ->] (Wu) to (bottomq);
\node at (1, 1.5) {$\restrict{W}{\bottom/}$};

\end{diagram}

Finally, suppose we pick $\{ s, u \}$ for patch candidates:

\begin{diagram}

\draw[rounded corners=4pt,fill=selected] (-2.5, 2.65) rectangle (-1.5, 2.35);
\draw[rounded corners=4pt,fill=selected] (1.5, 2.65) rectangle (2.5, 2.35);

\node (U) at (0, 3) {\textcolor{faded}{$U$}};
\node (V) at (-2, 1.5) {$V$};
\node (W) at (2, 1.5) {$W$};
\node (bottom) at (0, 0) {\textcolor{faded}{$\bottom/$}};

\draw[dashed,faded] (bottom) to (V);
\draw[dashed,faded] (bottom) to (W);
\draw[dashed,faded] (V) to (U);
\draw[dashed,faded] (W) to (U);

\draw (-2, 1.75) to (-2, 2.275);
\draw (-2, 3.1) ellipse (0.375cm and 0.875cm);
\node[dot, label=above:{$s$}] (Vs) at (-2, 2.5) {};
\node[dot, label=above:{$r$}] (Vr) at (-2, 3.25) {};

\draw (2, 1.75) to (2, 2.275);
\draw (2, 3.1) ellipse (0.375cm and 0.875cm);
\node[dot, label=above:{$u$}] (Wu) at (2, 2.5) {};
\node[dot, label=above:{$t$}] (Wt) at (2, 3.25) {};

\end{diagram}

These are not compatible. They do not agree on their overlap:

\begin{diagram}

\draw[rounded corners=4pt,fill=selected] (-2.5, 2.65) rectangle (-1.5, 2.35);
\draw[rounded corners=4pt,fill=selected] (1.5, 2.65) rectangle (2.5, 2.35);

\node (U) at (0, 3) {\textcolor{faded}{$U$}};
\node (V) at (-2, 1.5) {$V$};
\node (W) at (2, 1.5) {$W$};
\node (bottom) at (0, 0) {$\bottom/$};

\draw[dashed,faded] (bottom) to (V);
\draw[dashed,faded] (bottom) to (W);
\draw[dashed,faded] (V) to (U);
\draw[dashed,faded] (W) to (U);

\draw (-2, 1.75) to (-2, 2.275);
\draw (-2, 3.1) ellipse (0.375cm and 0.875cm);
\node[dot, label=above:{$s$}] (Vs) at (-2, 2.5) {};
\node[dot, label=above:{$r$}] (Vr) at (-2, 3.25) {};

\draw (2, 1.75) to (2, 2.275);
\draw (2, 3.1) ellipse (0.375cm and 0.875cm);
\node[dot, label=above:{$u$}] (Wu) at (2, 2.5) {};
\node[dot, label=above:{$c$}] (Wc) at (2, 3.25) {};

\draw (0, 0.25) to (0, 0.75);
\draw (0, 1.5) ellipse (0.375cm and 0.875cm);
\node[dot, label=above:{$p$}] (bottomp) at (0, 1) {};
\node[dot, label=above:{$q$}] (bottomq) at (0, 1.75) {};

\draw[arrow, ->, faded] (Vr) to (bottomq);
\draw[arrow, ->] (Vs) to (bottomp);
\node at (-1.25, 1.5) {$\restrict{V}{\bottom/}$};

\draw[arrow, ->, faded] (Wt) to (bottomp);
\draw[arrow, ->] (Wu) to (bottomq);
\node at (1, 1.5) {$\restrict{W}{\bottom/}$};

\end{diagram}

\end{Example}

% ----------------------------------------
\begin{Example}

Consider the empty cover. Since any selection of patch candidates for the empty cover is empty, compatibility is satisfied vacuously. 

\end{Example}


When selected patch candidates $s_{i}$, \ldots, $s_{k}$ across a cover of $U$ are compatible, we say those patches glue together if if there's a section $s$ in $F(U)$ that restricts down to exactly those patches.

% ----------------------------------------
 \begin{Definition}[Gluing]

Given a presheaf $F$ and a selection of compatible patch candidates $\{ s_{i} \}_{i \in I}$ for a cover $\{ U_{i} \}_{i \in I}$, $\{ s_{i} \}_{i \in I}$ \emph{glue} together only if there is a section $s \in F(U)$ that restricts down to $s_{i}$ on each fiber $F(U_{i})$ of the cover, i.e., only if $s$ is such that:
\[
\restrict{U}{U_{i}}(s) = s_{i}, \text{ for each } i \in I.
\]

\noindent
A selection of patches $\{ s_{i} \}_{i \in I}$ glues uniquely if there is one and only one such section $s \in F(U)$ that is glued from them.

\end{Definition}


% ----------------------------------------
\begin{Remark}

As a matter of terminology, if a section $s \in F(U)$ is glued from patches $\{ s_{i} \}_{i \in I}$, we say that $s$ is a global section of the cover, and each $s_{i}$ is a local section of the cover. We may also say variously that $s$ is a \emph{gluing} of those patches, that $s$ is \emph{composed} of those patches, that those patches \emph{compose} $s$, or that gluing those patches \emph{yields} $s$.
 
\end{Remark}


% ----------------------------------------
\begin{Example}
\label{ex:gluing}
 
Take the presheaf from \cref{ex:compatible-patch-candidates}, and consider the cover $\{ V, W \}$ again. Take the selection of patches $\{ s, t \}$, which are compatible because they agree on overlap:

\begin{diagram}

\draw[rounded corners=4pt,fill=selected] (-2.5, 2.65) rectangle (-1.5, 2.35);
\draw[rounded corners=4pt,fill=selected] (1.5, 3.4) rectangle (2.5, 3.1);

\node (U) at (0, 3) {\textcolor{faded}{$U$}};
\node (V) at (-2, 1.5) {$V$};
\node (W) at (2, 1.5) {$W$};
\node (bottom) at (0, 0) {$\bottom/$};

\draw[dashed,faded] (bottom) to (V);
\draw[dashed,faded] (bottom) to (W);
\draw[dashed,faded] (V) to (U);
\draw[dashed,faded] (W) to (U);

\draw[faded] (0, 3.25) to (0, 3.75);
\draw[faded] (0, 4.925) ellipse (0.425cm and 1.325cm);
\node[dot, label=above:{\textcolor{faded}{$a$}}, faded] (Ua) at (0, 4) {};
\node[dot, label=above:{\textcolor{faded}{$b$}}, faded] (Ub) at (0, 4.75) {};
\node[dot, label=above:{\textcolor{faded}{$c$}}, faded] (Uc) at (0, 5.5) {};

\draw (-2, 1.75) to (-2, 2.275);
\draw (-2, 3.1) ellipse (0.375cm and 0.875cm);
\node[dot, label=above:{$s$}] (Vs) at (-2, 2.5) {};
\node[dot, label=above:{$r$}] (Vr) at (-2, 3.25) {};

\draw (2, 1.75) to (2, 2.275);
\draw (2, 3.1) ellipse (0.375cm and 0.875cm);
\node[dot, label=above:{$u$}] (Wu) at (2, 2.5) {};
\node[dot, label=above:{$t$}] (Wt) at (2, 3.25) {};

\draw (0, 0.25) to (0, 0.75);
\draw (0, 1.5) ellipse (0.375cm and 0.875cm);
\node[dot, label=above:{$p$}] (bottomp) at (0, 1) {};
\node[dot, label=above:{\textcolor{faded}{$q$}}, faded] (bottomq) at (0, 1.75) {};

(\draw[arrow,->, faded] (Uc) to (Vr);
\draw[arrow, ->, faded] (Ub) to (Vr);
\draw[arrow, ->, faded] (Ua) to (Vs);
\node at (-1.25, 4.5) {\textcolor{faded}{$\restrict{U}{V}$}};

\draw[arrow, ->, faded] (Ua) to (Wu);
\draw[arrow, ->, faded] (Ub) to (Wu);
\draw[arrow, ->, faded] (Uc) to (Wt);
\node at (1.675, 4.5) {\textcolor{faded}{$\restrict{U}{W}$}};

\draw[arrow, ->, faded] (Vr) to (bottomq);
\draw[arrow, ->] (Vs) to (bottomp);
\node at (-1.25, 1.5) {$\restrict{V}{\bottom/}$};

\draw[arrow, ->, faded] (Wu) to (bottomq);
\draw[arrow, ->] (Wt) to (bottomp);
\node at (1, 1.5) {$\restrict{W}{\bottom/}$};

\end{diagram}

Even though $s$ and $t$ are compatible, they do not glue together, because there is no section in $F(U)$ that restricts down to them. Consider $a \in F(U)$ first. It restricts to $s \in F(V)$ on the left, but it does not restrict to $t \in F(W)$ on the right:

\begin{diagram}

\draw[rounded corners=4pt,fill=selected] (-2.5, 2.65) rectangle (-1.5, 2.35);
\draw[rounded corners=4pt,fill=selected] (1.5, 3.4) rectangle (2.5, 3.1);

\node (U) at (0, 3) {\textcolor{faded}{$U$}};
\node (V) at (-2, 1.5) {$V$};
\node (W) at (2, 1.5) {$W$};
\node (bottom) at (0, 0) {$\bottom/$};

\draw[dashed,faded] (bottom) to (V);
\draw[dashed,faded] (bottom) to (W);
\draw[dashed,faded] (V) to (U);
\draw[dashed,faded] (W) to (U);

\draw[faded] (0, 3.25) to (0, 3.75);
\draw[faded] (0, 4.925) ellipse (0.425cm and 1.325cm);
\node[dot, label=above:{$a$}] (Ua) at (0, 4) {};
\node[dot, label=above:{\textcolor{faded}{$b$}}, faded] (Ub) at (0, 4.75) {};
\node[dot, label=above:{\textcolor{faded}{$c$}}, faded] (Uc) at (0, 5.5) {};

\draw (-2, 1.75) to (-2, 2.275);
\draw (-2, 3.1) ellipse (0.375cm and 0.875cm);
\node[dot, label=above:{$s$}] (Vs) at (-2, 2.5) {};
\node[dot, label=above:{$r$}] (Vr) at (-2, 3.25) {};

\draw (2, 1.75) to (2, 2.275);
\draw (2, 3.1) ellipse (0.375cm and 0.875cm);
\node[dot, label=above:{$u$}] (Wu) at (2, 2.5) {};
\node[dot, label=above:{$t$}] (Wt) at (2, 3.25) {};

\draw (0, 0.25) to (0, 0.75);
\draw (0, 1.5) ellipse (0.375cm and 0.875cm);
\node[dot, label=above:{$p$}] (bottomp) at (0, 1) {};
\node[dot, label=above:{\textcolor{faded}{$q$}}, faded] (bottomq) at (0, 1.75) {};

\draw[arrow, ->, faded] (Vr) to (bottomq);
\draw[arrow, ->] (Vs) to (bottomp);
\node at (-1.25, 1.5) {$\restrict{V}{\bottom/}$};

\draw[arrow, ->, faded] (Wu) to (bottomq);
\draw[arrow, ->] (Wt) to (bottomp);
\node at (1, 1.5) {$\restrict{W}{\bottom/}$};

(\draw[arrow,->, faded] (Uc) to (Vr);
\draw[arrow, ->, faded] (Ub) to (Vr);
\draw[arrow, ->] (Ua) to (Vs);
\node at (-1.25, 4.5) {\textcolor{faded}{$\restrict{U}{V}$}};

\draw[arrow, ->] (Ua) to (Wu);
\draw[arrow, ->, faded] (Ub) to (Wu);
\draw[arrow, ->, faded] (Uc) to (Wt);
\node at (1.675, 4.5) {\textcolor{faded}{$\restrict{U}{W}$}};

\draw[ultra thick, wrong] (-0.3, 4.3) to (0.3, 3.6);
\draw[ultra thick, wrong] (0.3, 4.3) to (-0.3, 3.6);

\end{diagram}

As for $b \in F(U)$, it restricts to neither $s \in F(V)$  on the left nor $t \in F(W)$ on the right:

\begin{diagram}

\draw[rounded corners=4pt,fill=selected] (-2.5, 2.65) rectangle (-1.5, 2.35);
\draw[rounded corners=4pt,fill=selected] (1.5, 3.4) rectangle (2.5, 3.1);

\node (U) at (0, 3) {\textcolor{faded}{$U$}};
\node (V) at (-2, 1.5) {$V$};
\node (W) at (2, 1.5) {$W$};
\node (bottom) at (0, 0) {$\bottom/$};

\draw[dashed,faded] (bottom) to (V);
\draw[dashed,faded] (bottom) to (W);
\draw[dashed,faded] (V) to (U);
\draw[dashed,faded] (W) to (U);

\draw[faded] (0, 3.25) to (0, 3.75);
\draw[faded] (0, 4.925) ellipse (0.425cm and 1.325cm);
\node[dot, label=above:{\textcolor{faded}{$a$}}, faded] (Ua) at (0, 4) {};
\node[dot, label=above:{$b$}] (Ub) at (0, 4.75) {};
\node[dot, label=above:{\textcolor{faded}{$c$}}, faded] (Uc) at (0, 5.5) {};

\draw (-2, 1.75) to (-2, 2.275);
\draw (-2, 3.1) ellipse (0.375cm and 0.875cm);
\node[dot, label=above:{$s$}] (Vs) at (-2, 2.5) {};
\node[dot, label=above:{$r$}] (Vr) at (-2, 3.25) {};

\draw (2, 1.75) to (2, 2.275);
\draw (2, 3.1) ellipse (0.375cm and 0.875cm);
\node[dot, label=above:{$u$}] (Wu) at (2, 2.5) {};
\node[dot, label=above:{$t$}] (Wt) at (2, 3.25) {};

\draw (0, 0.25) to (0, 0.75);
\draw (0, 1.5) ellipse (0.375cm and 0.875cm);
\node[dot, label=above:{$p$}] (bottomp) at (0, 1) {};
\node[dot, label=above:{\textcolor{faded}{$q$}}, faded] (bottomq) at (0, 1.75) {};

\draw[arrow, ->, faded] (Vr) to (bottomq);
\draw[arrow, ->] (Va) to (bottomp);
\node at (-1.25, 1.5) {$\restrict{V}{\bottom/}$};

\draw[arrow, ->, faded] (Wu) to (bottomq);
\draw[arrow, ->] (Wt) to (bottomp);
\node at (1, 1.5) {$\restrict{W}{\bottom/}$};

(\draw[arrow,->, faded] (Uc) to (Vr);
\draw[arrow, ->] (Ub) to (Vr);
\draw[arrow, ->, faded] (Ua) to (Vs);
\node at (-1.25, 4.5) {\textcolor{faded}{$\restrict{U}{V}$}};

\draw[arrow, ->, faded] (Ua) to (Wu);
\draw[arrow, ->] (Ub) to (Wu);
\draw[arrow, ->, faded] (Uc) to (Wt);
\node at (1.675, 4.5) {\textcolor{faded}{$\restrict{U}{W}$}};

\draw[ultra thick, wrong] (-0.3, 5.2) to (0.3, 4.6);
\draw[ultra thick, wrong] (0.3, 5.2) to (-0.3, 4.6);

\end{diagram}

Finally, $c \in F(U)$ restricts to $t \in F(W)$ on the right, but not to $s \in F(V)$ on the left:

\begin{diagram}

\draw[rounded corners=4pt,fill=selected] (-2.5, 2.65) rectangle (-1.5, 2.35);
\draw[rounded corners=4pt,fill=selected] (1.5, 3.4) rectangle (2.5, 3.1);

\node (U) at (0, 3) {\textcolor{faded}{$U$}};
\node (V) at (-2, 1.5) {$V$};
\node (W) at (2, 1.5) {$W$};
\node (bottom) at (0, 0) {$\bottom/$};

\draw[dashed,faded] (bottom) to (V);
\draw[dashed,faded] (bottom) to (W);
\draw[dashed,faded] (V) to (U);
\draw[dashed,faded] (W) to (U);

\draw[faded] (0, 3.25) to (0, 3.75);
\draw[faded] (0, 4.925) ellipse (0.425cm and 1.325cm);
\node[dot, label=above:{\textcolor{faded}{$a$}}, faded] (Ua) at (0, 4) {};
\node[dot, label=above:{\textcolor{faded}{$b$}}, faded] (Ub) at (0, 4.75) {};
\node[dot, label=above:{$c$}] (Uc) at (0, 5.5) {};

\draw (-2, 1.75) to (-2, 2.275);
\draw (-2, 3.1) ellipse (0.375cm and 0.875cm);
\node[dot, label=above:{$s$}] (Vs) at (-2, 2.5) {};
\node[dot, label=above:{$r$}] (Vr) at (-2, 3.25) {};

\draw (2, 1.75) to (2, 2.275);
\draw (2, 3.1) ellipse (0.375cm and 0.875cm);
\node[dot, label=above:{$u$}] (Wu) at (2, 2.5) {};
\node[dot, label=above:{$t$}] (Wt) at (2, 3.25) {};

\draw (0, 0.25) to (0, 0.75);
\draw (0, 1.5) ellipse (0.375cm and 0.875cm);
\node[dot, label=above:{$p$}] (bottomp) at (0, 1) {};
\node[dot, label=above:{\textcolor{faded}{$q$}}, faded] (bottomq) at (0, 1.75) {};

\draw[arrow, ->, faded] (Vr) to (bottomq);
\draw[arrow, ->] (Vs) to (bottomp);
\node at (-1.25, 1.5) {$\restrict{V}{\bottom/}$};

\draw[arrow, ->, faded] (Wu) to (bottomq);
\draw[arrow, ->] (Wt) to (bottomp);
\node at (1, 1.5) {$\restrict{W}{\bottom/}$};

(\draw[arrow,->] (Uc) to (Vr);
\draw[arrow, ->, faded] (Ub) to (Vr);
\draw[arrow, ->, faded] (Ua) to (Vs);
\node at (-1.25, 4.5) {\textcolor{faded}{$\restrict{U}{V}$}};

\draw[arrow, ->, faded] (Ua) to (Wu);
\draw[arrow, ->, faded] (Ub) to (Wu);
\draw[arrow, ->] (Uc) to (Wt);
\node at (1.675, 4.5) {\textcolor{faded}{$\restrict{U}{W}$}};

\draw[ultra thick, wrong] (-0.3, 5.9) to (0.3, 5.3);
\draw[ultra thick, wrong] (0.3, 5.9) to (-0.3, 5.3);

\end{diagram}

Thus, none of $a$, $b$, or $c$ in $F(U)$ are glued from $\{ s, t \}$, because none of them decompose into $s$ on the left and $t$ on the right.

Now suppose we pick $\{ r, u \}$ for patch candidates. These do glue together, because there is a section in $F(U)$ (namely $b \in F(U)$) that restricts down to $r \in F(V)$ on the left and $u \in F(W)$ on the right:

\begin{diagram}

\draw[rounded corners=4pt,fill=selected] (-2.5, 3.4) rectangle (-1.5, 3.1);
\draw[rounded corners=4pt,fill=selected] (1.5, 2.65) rectangle (2.5, 2.35);
\draw[rounded corners=4pt,fill=selected2] (-0.5, 4.9) rectangle (0.5, 4.6);

\node (U) at (0, 3) {\textcolor{faded}{$U$}};
\node (V) at (-2, 1.5) {$V$};
\node (W) at (2, 1.5) {$W$};
\node (bottom) at (0, 0) {$\bottom/$};

\draw[dashed,faded] (bottom) to (V);
\draw[dashed,faded] (bottom) to (W);
\draw[dashed,faded] (V) to (U);
\draw[dashed,faded] (W) to (U);

\draw[faded] (0, 3.25) to (0, 3.75);
\draw[faded] (0, 4.925) ellipse (0.425cm and 1.325cm);
\node[dot, label=above:{\textcolor{faded}{$a$}}, faded] (Ua) at (0, 4) {};
\node[dot, label=above:{$b$}] (Ub) at (0, 4.75) {};
\node[dot, label=above:{\textcolor{faded}{$c$}}, faded] (Uc) at (0, 5.5) {};

\draw (-2, 1.75) to (-2, 2.275);
\draw (-2, 3.1) ellipse (0.375cm and 0.875cm);
\node[dot, label=above:{$s$}] (Vs) at (-2, 2.5) {};
\node[dot, label=above:{$r$}] (Vr) at (-2, 3.25) {};

\draw (2, 1.75) to (2, 2.275);
\draw (2, 3.1) ellipse (0.375cm and 0.875cm);
\node[dot, label=above:{$u$}] (Wu) at (2, 2.5) {};
\node[dot, label=above:{$t$}] (Wt) at (2, 3.25) {};

\draw (0, 0.25) to (0, 0.75);
\draw (0, 1.5) ellipse (0.375cm and 0.875cm);
\node[dot, label=above:{\textcolor{faded}{$p$}}, faded] (bottomp) at (0, 1) {};
\node[dot, label=above:{$q$}] (bottomq) at (0, 1.75) {};

\draw[arrow, ->, faded] (Uc) to (Vr);
\draw[arrow, ->] (Ub) to (Vr);
\draw[arrow, ->, faded] (Ua) to (Vs);
\node at (-1.25, 4.5) {\textcolor{faded}{$\restrict{U}{V}$}};

\draw[arrow, ->, faded] (Ua) to (Wu);
\draw[arrow, ->] (Ub) to (Wu);
\draw[arrow, ->, faded] (Uc) to (Wt);
\node at (1.675, 4.5) {\textcolor{faded}{$\restrict{U}{W}$}};

\draw[arrow, ->] (Vr) to (bottomq);
\draw[arrow, ->, faded] (Vs) to (bottomp);
\node at (-1.25, 1.5) {$\restrict{V}{\bottom/}$};

\draw[arrow, ->, faded] (Wt) to (bottomp);
\draw[arrow, ->] (Wu) to (bottomq);
\node at (1, 1.5) {$\restrict{W}{\bottom/}$};

\end{diagram}
 
\end{Example}

% ----------------------------------------
\begin{Example}
\label{ex:robot}

Consider an example that glues together behaviors. Imagine a toy robot that looks something like a small tank: it has tracks on the left and right sides, and the two tracks are connected by a single drive controller. The controller either drives at a constant speed, or it sits idle. When it drives, it turns both tracks at the same speed.

Let's represent the robot as a locale. Let $LT$ and $RT$ be the left and right track assemblies respectively, let $D$ be the drive controller that is shared by $LT$ and $RT$, and let $R$ be the whole robot (the join of $LT$ and $RT$). As a picture:


\begin{diagram}

\node (R) at (0, 3) {$R$};
\node (LT) at (-2, 1.5) {$LT$};
\node (RT) at (2, 1.5) {$RT$};
\node (D) at (0, 0) {$D$};
\node (bottom) at (0, -1) {$\bottom/$};

\draw[] (D) to (bottom);
\draw[] (D) to (LT);
\draw[] (D) to (RT);
\draw[] (LT) to (R);
\draw[] (RT) to (R);

\end{diagram}

\noindent
For a presheaf, let's assign to each region the behaviors that are locally observable at that region:

\begin{itemize}

\item The drive controller $D$ can either $drive$ or sit $idle$. 
\item The left track assembly can either $move_{L}$ or $stand\mhyphen still_{L}$.
\item The right track assembly can also either $move_{R}$ or $stand\mhyphen still_{R}$.
\item The entire robot can either move $forward$ or $sit$ stationary.
\item For the fiber over $\bottom/$, where there are no regions that could carry any behaviors to begin with, assign the special symbol zero.

\end{itemize}

\noindent
In a picture:

\begin{diagram}

\node (R) at (0, 3) {$R$};
\node (LT) at (-2, 1.5) {$LT$};
\node (RT) at (2, 1.5) {$RT$};
\node (D) at (0, 0) {$D$};
\node (bottom) at (0, -1.5) {$\bottom/$};

\draw[dashed] (D) to (bottom);
\draw[dashed] (D) to (LT);
\draw[dashed] (D) to (RT);
\draw[dashed] (LT) to (R);
\draw[dashed] (RT) to (R);

\draw (0, 3.25) to (0, 3.75);
\draw (0, 4.5) ellipse (0.75cm and 1cm);
\node[dot, label=above:{\small{$sit$}}] (sit) at (0, 4.75) {};
\node[dot, label=above:{\small{$forward$}}] (move) at (0, 4) {};

\draw (-2, 1.75) to (-2, 2.275);
\draw (-2.75, 3) ellipse (1.25cm and 0.8cm);
\node[dot, label=left:{\small{$move_{L}$}}] (left-turn) at (-2, 2.6) {};
\node[dot, label=left:{\small{$stand\mhyphen still_{L}$}}] (left-stand) at (-2, 3.25) {};

\draw (2, 1.75) to (2, 2.275);
\draw (2.75, 3) ellipse (1.25cm and 0.8cm);
\node[dot, label=right:{\small{$move_{R}$}}] (right-turn) at (2, 2.6) {};
\node[dot, label=right:{\small{$stand\mhyphen still_{R}$}}] (right-stand) at (2, 3.25) {};

\draw (0, 0.25) to (0, 0.5);
\draw (0, 1.5) ellipse (0.75cm and 1cm);
\node[dot, label=below:{\small{$drive$}}] (drive) at (0, 1.25) {};
\node[dot, label=below:{\small{$idle$}}] (idle) at (0, 2) {};

\draw (0.25, -1.5) to (0.65, -1.1);
\draw (0.75, -0.65) ellipse (0.35cm and 0.5cm);
\node[dot, label=below:{\small{$0$}}] (asterisk) at (0.75, -0.575) {};

\end{diagram}

For the restriction maps, let's say that they restrict the observable behavior of a larger region to the observable behavior of the smaller region. For instance, if you are observing the whole robot moving forward ($forward$), and you then ``zoom in'' on the left track assembly, you'll see those tracks rotating ($move_{L}$).

\begin{itemize}

\item $\restrict{R}{LT}(sit) = stand\mhyphen still_{L}$, $\restrict{R}{LT}(forward) = move_{L}$.
\item $\restrict{R}{RT}(sit) = stand\mhyphen still_{R}$, $\restrict{R}{RT}(forward) = move_{R}$.
\item $\restrict{LT}{D}(stand\mhyphen still_{L}) = idle$, $\restrict{LT}{D}(move_{L}) = drive$.
\item $\restrict{RT}{D}(stand\mhyphen still_{R}) = idle$, $\restrict{RT}{D}(move_{R}) = drive$.
\item $\restrict{D}{\bottom/}(idle) = \restrict{D}{\bottom/}(drive) = 0$.

\end{itemize}

\noindent
In a picture:

\begin{diagram}

\node (R) at (0, 3) {$R$};
\node (LT) at (-2, 1.5) {$LT$};
\node (RT) at (2, 1.5) {$RT$};
\node (D) at (0, 0) {$D$};
\node (bottom) at (0, -1.5) {$\bottom/$};

\draw[dashed] (D) to (bottom);
\draw[dashed] (D) to (LT);
\draw[dashed] (D) to (RT);
\draw[dashed] (LT) to (R);
\draw[dashed] (RT) to (R);

\draw (0, 3.25) to (0, 3.75);
\draw (0, 4.5) ellipse (0.75cm and 1cm);
\node[dot, label=above:{\small{$sit$}}] (sit) at (0, 4.75) {};
\node[dot, label=above:{\small{$forward$}}] (move) at (0, 4) {};

\draw (-2, 1.75) to (-2, 2.275);
\draw (-2.75, 3) ellipse (1.25cm and 0.8cm);
\node[dot, label=left:{\small{$move_{L}$}}] (left-turn) at (-2, 2.6) {};
\node[dot, label=left:{\small{$stand\mhyphen still_{L}$}}] (left-stand) at (-2, 3.25) {};

\draw (2, 1.75) to (2, 2.275);
\draw (2.75, 3) ellipse (1.25cm and 0.8cm);
\node[dot, label=right:{\small{$move_{R}$}}] (right-turn) at (2, 2.6) {};
\node[dot, label=right:{\small{$stand\mhyphen still_{R}$}}] (right-stand) at (2, 3.25) {};

\draw (0, 0.25) to (0, 0.5);
\draw (0, 1.5) ellipse (0.75cm and 1cm);
\node[dot, label=below:{\small{$drive$}}] (drive) at (0, 1.25) {};
\node[dot, label=below:{\small{$idle$}}] (idle) at (0, 2) {};

\draw (0.25, -1.5) to (0.65, -1.1);
\draw (0.75, -0.65) ellipse (0.35cm and 0.5cm);
\node[dot, label=below:{\small{$0$}}] (asterisk) at (0.75, -0.575) {};

\draw[arrow, ->] (move) to (left-turn);
\draw[arrow, ->] (sit) to (left-stand);
\node at (-1.65, 4.375) {$\restrict{R}{LT}$};

\draw[arrow, ->] (move) to (right-turn);
\draw[arrow, ->] (sit) to (right-stand);
\node at (1.65, 4.375) {$\restrict{R}{RT}$};

\draw[arrow, ->] (left-turn) to (drive);
\draw[arrow, ->] (left-stand) to (idle);
\node at (-1.25, 1.5) {$\restrict{LT}{D}$};

\draw[arrow, ->] (right-turn) to (drive);
\draw[arrow, ->] (right-stand) to (idle);
\node at (1.25, 1.5) {$\restrict{RT}{D}$};

\draw[arrow, ->] (idle) to[out=325, in=80] (asterisk);
\draw[arrow, ->] (drive) to[out=335, in=85] (asterisk);
\node at (1.275, 0.25) {$\restrict{D}{\bottom/}$};

\end{diagram}

Now take the cover $\{ LT, RT \}$ of $R$. The patch candidates $\{ move_{L}, move_{R} \}$ are compatible, because they agree on overlap (they both restrict down to $drive$). But they also glue uniquely, yielding $forward$. In other words, the robot's forward motion is patched together precisely from the two pieces of its cover, namely the left tracks rotating ($move_{L}$) and the right tracks rotating ($move_{R}$).

Similarly, the Robot's sitting still ($sit$) behavior is also glued from the two pieces of its cover, namely the left track assembly standing still ($stand\mhyphen still_{L}$) and the right track assembly standing still ($stand\mhyphen still_{R}$).

Thus, there are two global sections of $R$'s behavior: moving forwards (patched together from its left and right motions), or standing still (patched together from its left and right lack of motion). 

\end{Example}


%%%%%%%%%%%%%%%%%%%%%%%%%%%%%%%%%%%%%%%%%%
\subsection{Monopresheaves}

\noindent
In a presheaf, compatible patch candidates can be glued together to form sections that span multiple fibers. However, nothing said so far prevents there being multiple gluings from the same patch candidates. In other words, nothing requires gluings to be extensional. 

If we want to work only with extensional gluings, then we can impose a restriction that says gluings must be unique: i.e., if a selection of patch candidates can glue, they form at most one gluing. Presheaves where this obtains are called monopresheaves.

% ----------------------------------------
\begin{Definition}[Monopresheaf]
 
A presheaf $F$ is a \emph{monopresheaf} iff it satisfies the following gluing-uniqueness condition:

\begin{enumerate}

\item [(G1)] For every cover $\{ U_{i} \}_{i \in I}$ of a region $U$ and every selection of compatible patch candidates $\{ s_{i} \}_{i \in I}$ for that cover, if there is a gluing $s \in F(U)$ of $\{ s_{i} \}_{i \in I}$, then it is unique.

\end{enumerate}

\noindent
Equivalently, given $s, t \in F(U)$: 

\begin{enumerate}

\item [(G1$^{\ast}$)] For every cover $\{ U_{i} \}_{i \in I}$ of $U$, if $\restrict{U}{U_{i}}(s) = \restrict{U}{U_{i}}(t)$ for each $U_{i}$, then $s = t$.

\end{enumerate}
 
\end{Definition}

\begin{Remark}

Monopresheaves are also called \emph{separated presheaves}. The ``mono'' part of the name comes from category theory: every joint restriction to a covering family is a monomorphism, i.e., there can be at most one section restricting to a given selection of patch candidates.

\end{Remark}


% ----------------------------------------
\begin{Example}
\label{ex:non-monopresheaf}
 
By way of counter-example, consider the following presheaf:

\begin{diagram}

\node (U) at (0, 3) {$U$};
\node (V) at (-2, 1.5) {$V$};
\node (W) at (2, 1.5) {$W$};
\node (bottom) at (0, 0) {$\bottom/$};

\draw[dashed] (bottom) to (V);
\draw[dashed] (bottom) to (W);
\draw[dashed] (V) to (U);
\draw[dashed] (W) to (U);

\draw (0, 3.25) to (0, 3.75);
\draw (0, 4.925) ellipse (0.425cm and 1.325cm);
\node[dot, label=above:{$a$}] (Ua) at (0, 4) {};
\node[dot, label=above:{$b$}] (Ub) at (0, 4.75) {};
\node[dot, label=above:{$c$}] (Uc) at (0, 5.5) {};

\draw (-2, 1.75) to (-2, 2.275);
\draw (-2, 3.1) ellipse (0.375cm and 0.875cm);
\node[dot, label=above:{$s$}] (Vs) at (-2, 2.5) {};
\node[dot, label=above:{$r$}] (Vr) at (-2, 3.25) {};

\draw (2, 1.75) to (2, 2.275);
\draw (2, 3.1) ellipse (0.375cm and 0.875cm);
\node[dot, label=above:{$u$}] (Wu) at (2, 2.5) {};
\node[dot, label=above:{$t$}] (Wt) at (2, 3.25) {};

\draw (0, 0.25) to (0, 0.75);
\draw (0, 1.5) ellipse (0.375cm and 0.875cm);
\node[dot, label=above:{$p$}] (bottomp) at (0, 1) {};
\node[dot, label=above:{$q$}] (bottomq) at (0, 1.75) {};

(\draw[arrow,->] (Uc) to (Vr);
\draw[arrow, ->] (Ub) to (Vr);
\draw[arrow, ->] (Ua) to (Vs);
\node at (-1.5, 4.5) {$\restrict{U}{V}$};

\draw[arrow, ->] (Ua) to (Wu);
\draw[arrow, ->] (Ub) to (Wt);
\draw[arrow, ->] (Uc) to (Wt);
\node at (1.675, 4.5) {$\restrict{U}{W}$};

\draw[arrow, ->] (Vr) to (bottomq);
\draw[arrow, ->] (Vs) to (bottomp);
\node at (-1.25, 1.4) {$\restrict{V}{\bottom/}$};

\draw[arrow, ->] (Wt) to (bottomq);
\draw[arrow, ->] (Wu) to (bottomp);
\node at (1.25, 1.4) {$\restrict{W}{\bottom/}$};

\end{diagram}

Take the cover $\{ V, W \}$ and selection of patch candidates $\{ r, t \}$, which are compatible because they agree on overlap:

\begin{diagram}

\draw[rounded corners=4pt,fill=selected] (-2.5, 3.4) rectangle (-1.5, 3.1);
\draw[rounded corners=4pt,fill=selected] (1.5, 3.4) rectangle (2.5, 3.1);

\node (U) at (0, 3) {\textcolor{faded}{$U$}};
\node (V) at (-2, 1.5) {$V$};
\node (W) at (2, 1.5) {$W$};
\node (bottom) at (0, 0) {$\bottom/$};

\draw[dashed,faded] (bottom) to (V);
\draw[dashed,faded] (bottom) to (W);
\draw[dashed,faded] (V) to (U);
\draw[dashed,faded] (W) to (U);

\draw[faded] (0, 3.25) to (0, 3.75);
\draw[faded] (0, 4.925) ellipse (0.425cm and 1.325cm);
\node[dot, label=above:{\textcolor{faded}{$a$}}, faded] (Ua) at (0, 4) {};
\node[dot, label=above:{\textcolor{faded}{$b$}}, faded] (Ub) at (0, 4.75) {};
\node[dot, label=above:{\textcolor{faded}{$c$}}, faded] (Uc) at (0, 5.5) {};

\draw (-2, 1.75) to (-2, 2.275);
\draw (-2, 3.1) ellipse (0.375cm and 0.875cm);
\node[dot, label=above:{$s$}] (Vs) at (-2, 2.5) {};
\node[dot, label=above:{$r$}] (Vr) at (-2, 3.25) {};

\draw (2, 1.75) to (2, 2.275);
\draw (2, 3.1) ellipse (0.375cm and 0.875cm);
\node[dot, label=above:{$u$}] (Wu) at (2, 2.5) {};
\node[dot, label=above:{$t$}] (Wt) at (2, 3.25) {};

\draw (0, 0.25) to (0, 0.75);
\draw (0, 1.5) ellipse (0.375cm and 0.875cm);
\node[dot, label=above:{\textcolor{faded}{$p$}}, faded] (bottomp) at (0, 1) {};
\node[dot, label=above:{$q$}] (bottomq) at (0, 1.75) {};

(\draw[arrow,->, faded] (Uc) to (Vr);
\draw[arrow, ->, faded] (Ub) to (Vr);
\draw[arrow, ->, faded] (Ua) to (Vs);
\node at (-1.5, 4.5) {\textcolor{faded}{$\restrict{U}{V}$}};

\draw[arrow, ->, faded] (Ua) to (Wu);
\draw[arrow, ->, faded] (Ub) to (Wt);
\draw[arrow, ->, faded] (Uc) to (Wt);
\node at (1.675, 4.5) {\textcolor{faded}{$\restrict{U}{W}$}};

\draw[arrow, ->] (Vr) to (bottomq);
\draw[arrow, ->, faded] (Vs) to (bottomp);
\node at (-1.25, 1.4) {$\restrict{V}{\bottom/}$};

\draw[arrow, ->] (Wt) to (bottomq);
\draw[arrow, ->, faded] (Wu) to (bottomp);
\node at (1.25, 1.4) {$\restrict{W}{\bottom/}$};

\end{diagram}

Do $r$ and $t$ glue? That is to say, is there a section in $F(U)$ that decomposes exactly to $r$ and $t$? In this case, $b \in F(U)$ is a gluing of $r$ and $t$:

\begin{diagram}

\draw[rounded corners=4pt,fill=selected] (-2.5, 3.4) rectangle (-1.5, 3.1);
\draw[rounded corners=4pt,fill=selected] (1.5, 3.4) rectangle (2.5, 3.1);
\draw[rounded corners=4pt,fill=selected2] (-0.5, 4.9) rectangle (0.5, 4.6);

\node (U) at (0, 3) {\textcolor{faded}{$U$}};
\node (V) at (-2, 1.5) {$V$};
\node (W) at (2, 1.5) {$W$};
\node (bottom) at (0, 0) {$\bottom/$};

\draw[dashed,faded] (bottom) to (V);
\draw[dashed,faded] (bottom) to (W);
\draw[dashed,faded] (V) to (U);
\draw[dashed,faded] (W) to (U);

\draw[faded] (0, 3.25) to (0, 3.75);
\draw[faded] (0, 4.925) ellipse (0.425cm and 1.325cm);
\node[dot, label=above:{\textcolor{faded}{$a$}}, faded] (Ua) at (0, 4) {};
\node[dot, label=above:{$b$}] (Ub) at (0, 4.75) {};
\node[dot, label=above:{\textcolor{faded}{$c$}}, faded] (Uc) at (0, 5.5) {};

\draw (-2, 1.75) to (-2, 2.275);
\draw (-2, 3.1) ellipse (0.375cm and 0.875cm);
\node[dot, label=above:{$s$}] (Vs) at (-2, 2.5) {};
\node[dot, label=above:{$r$}] (Vr) at (-2, 3.25) {};

\draw (2, 1.75) to (2, 2.275);
\draw (2, 3.1) ellipse (0.375cm and 0.875cm);
\node[dot, label=above:{$u$}] (Wu) at (2, 2.5) {};
\node[dot, label=above:{$t$}] (Wt) at (2, 3.25) {};

\draw (0, 0.25) to (0, 0.75);
\draw (0, 1.5) ellipse (0.375cm and 0.875cm);
\node[dot, label=above:{\textcolor{faded}{$p$}}, faded] (bottomp) at (0, 1) {};
\node[dot, label=above:{$q$}] (bottomq) at (0, 1.75) {};

(\draw[arrow,->, faded] (Uc) to (Vr);
\draw[arrow, ->] (Ub) to (Vr);
\draw[arrow, ->, faded] (Ua) to (Vs);
\node at (-1.5, 4.5) {\textcolor{faded}{$\restrict{U}{V}$}};

\draw[arrow, ->, faded] (Ua) to (Wu);
\draw[arrow, ->] (Ub) to (Wt);
\draw[arrow, ->, faded] (Uc) to (Wt);
\node at (1.675, 4.5) {\textcolor{faded}{$\restrict{U}{W}$}};

\draw[arrow, ->] (Vr) to (bottomq);
\draw[arrow, ->, faded] (Vs) to (bottomp);
\node at (-1.25, 1.4) {$\restrict{V}{\bottom/}$};

\draw[arrow, ->] (Wt) to (bottomq);
\draw[arrow, ->, faded] (Wu) to (bottomp);
\node at (1.25, 1.4) {$\restrict{W}{\bottom/}$};

\end{diagram}

However, $b$ is not a unique gluing, since $c$ is also a gluing of $r$ and $t$:

\begin{diagram}

\draw[rounded corners=4pt,fill=selected] (-2.5, 3.4) rectangle (-1.5, 3.1);
\draw[rounded corners=4pt,fill=selected] (1.5, 3.4) rectangle (2.5, 3.1);
\draw[rounded corners=4pt,fill=selected2] (-0.5, 5.65) rectangle (0.5, 5.35);

\node (U) at (0, 3) {\textcolor{faded}{$U$}};
\node (V) at (-2, 1.5) {$V$};
\node (W) at (2, 1.5) {$W$};
\node (bottom) at (0, 0) {$\bottom/$};

\draw[dashed,faded] (bottom) to (V);
\draw[dashed,faded] (bottom) to (W);
\draw[dashed,faded] (V) to (U);
\draw[dashed,faded] (W) to (U);

\draw[faded] (0, 3.25) to (0, 3.75);
\draw[faded] (0, 4.925) ellipse (0.425cm and 1.325cm);
\node[dot, label=above:{\textcolor{faded}{$a$}}, faded] (Ua) at (0, 4) {};
\node[dot, label=above:{\textcolor{faded}{$b$}}, faded] (Ub) at (0, 4.75) {};
\node[dot, label=above:{$c$}] (Uc) at (0, 5.5) {};

\draw (-2, 1.75) to (-2, 2.275);
\draw (-2, 3.1) ellipse (0.375cm and 0.875cm);
\node[dot, label=above:{$s$}] (Vs) at (-2, 2.5) {};
\node[dot, label=above:{$r$}] (Vr) at (-2, 3.25) {};

\draw (2, 1.75) to (2, 2.275);
\draw (2, 3.1) ellipse (0.375cm and 0.875cm);
\node[dot, label=above:{$u$}] (Wu) at (2, 2.5) {};
\node[dot, label=above:{$t$}] (Wt) at (2, 3.25) {};

\draw (0, 0.25) to (0, 0.75);
\draw (0, 1.5) ellipse (0.375cm and 0.875cm);
\node[dot, label=above:{\textcolor{faded}{$p$}}, faded] (bottomp) at (0, 1) {};
\node[dot, label=above:{$q$}] (bottomq) at (0, 1.75) {};

(\draw[arrow,->] (Uc) to (Vr);
\draw[arrow, ->, faded] (Ub) to (Vr);
\draw[arrow, ->, faded] (Ua) to (Vs);
\node at (-1.5, 4.5) {\textcolor{faded}{$\restrict{U}{V}$}};

\draw[arrow, ->, faded] (Ua) to (Wu);
\draw[arrow, ->, faded] (Ub) to (Wt);
\draw[arrow, ->] (Uc) to (Wt);
\node at (1.675, 4.5) {\textcolor{faded}{$\restrict{U}{W}$}};

\draw[arrow, ->] (Vr) to (bottomq);
\draw[arrow, ->, faded] (Vs) to (bottomp);
\node at (-1.25, 1.4) {$\restrict{V}{\bottom/}$};

\draw[arrow, ->] (Wt) to (bottomq);
\draw[arrow, ->, faded] (Wu) to (bottomp);
\node at (1.25, 1.4) {$\restrict{W}{\bottom/}$};

\end{diagram}

Thus, this is not a monopresheaf, since gluable patch candidates don't glue uniquely. 

\end{Example}


% ----------------------------------------
\begin{Example}

On the other hand, the presheaf from \cref{ex:compatible-patch-candidates} is a monopresheaf, for whenever patch candidates glue together in that presheaf, they do so uniquely.

\end{Example}


%%%%%%%%%%%%%%%%%%%%%%%%%%%%%%%%%%%%%%%%%%
\subsection{Sheaves}
\label{sec:sheaves}

\noindent
The definition of a monopresheaf requires only that if compatible patch candidates glue, they do so uniquely. It does not require that compatible patch candidates always do glue together. Patch candidates in a monopresheaf need not glue.

If we want to work with monopresheaves where all gluable patch candidates do in fact glue together, then we can work with sheaves. A sheaf is a monopresheaf that satisfies an extra existence requirement: whenever patch candidates \emph{can} glue, they \emph{do} glue.

 
% ----------------------------------------
\begin{Definition}[Sheaf]
 
A monopresheaf $F$ is a \emph{sheaf} iff it satisfies the following gluing-existence condition:

\begin{enumerate}

\item [(G2)] For every cover $\{ U_{i} \}_{i \in I}$ of a region $U$ and every selection of patch candidates $\{ s_{i} \}_{i \in I}$ for that cover, if $\{ s_{i} \}_{i \in I}$ are compatible, then there exists a unique gluing $s \in F(U)$ of $\{ s_{i} \}_{i \in I}$.

\end{enumerate}
 
\end{Definition}

% ----------------------------------------
\begin{Example}

The presheaf from \cref{ex:compatible-patch-candidates} fails to be sheaf, because as we saw in \cref{ex:gluing}, there is a compatible selection of patch candidates (namely, $\{ s, t \}$) which fails to glue. To be a sheaf, every compatible selection of patch candidates must glue.

\end{Example}

There is a subtlety regarding what sheaves look like over the least element of a locale. Note that the gluing condition is formulated as an implication. That is to say, it says that, for every cross-section of patch candidates, \emph{if} that cross-section can glue, \emph{then} it glues in exactly one way. 

Next, consider the fact that the cover over the least region of a locale is an empty cover. Since there are no patch candidates that need to be checked for compatibility, there is nothing that needs to be done to get a ``selection of gluable patch candidates.'' Hence, the antecedent of the gluing condition is satisfied vacuously over the least element of the locale.

But since the empty cover satisfies the antecedent of the gluing condition vacuously, it follows that if a presheaf is to qualify as a sheaf, it must ensure that the consequent is satisfied over the empty cover as well. In other words, it must assign a unique glued section (a singleton set) to the least region of the locale. So, even though a \emph{presheaf} or \emph{monopresheaf} may assign a larger set of data to the least element of a locale, a \emph{sheaf} always assigns a singleton to that region.


% ----------------------------------------
\begin{Example}
\label{ex:robot-sheaf}

The presheaf from \cref{ex:robot} is a sheaf. Note that the bottom fiber is a singleton. This ensures that \emph{all} gluable selections of patch candidates (including the empty one) glue uniquely.

\end{Example}

If we consider presheaves, monopresheaves, and sheaves together, we see that we have a hierarchy of increasingly strict gluing requirements. (1) Presheaves have no gluing requirements. (2) Monopresheaves have a uniqueness requirement: gluings need not exist, but when they do, they are unique. (3) Sheaves have both a uniqueness and an existence requirement: gluings exist whenever possible, and they are unique. 


%%%%%%%%%%%%%%%%%%%%%%%%%%%%%%%%%%%%%%%%%%
\subsection{The bottom fiber}
\label{sec:the-bottom-fiber}

\noindent
The bottom element of a locale represents no regions at all. Thus, it plays a special role. Since it represents the \emph{absence} of any regions, the data that we assign to its fiber is of a different kind than the data we assign to other fibers. 

The other fibers sit over genuine regions in the parts space, so the data we assign to them plays a kind of ontological role: it's the ``stuff'' that occupies that region. By contrast, the fiber over $\bottom/$ cannot play this role. Since $\bottom/$ represents no region at all, its fiber cannot represent material occupancy. Instead, it plays a structural role.

Since any compatibility check between patch candidates ultimately factors through restriction maps that ultimately land in the fiber over $\bottom/$, agreement at $\bottom/$ functions as a final anchor point. Given this fact, we can make a general observation: there are as many kinds or modes of gluing as there are anchor points over $\bottom/$. 

If there is a single point in the fiber over $\bottom/$, then there will be only one kind of gluing that occurs throughout the presheaf. This is most evident in a sheaf, which as we saw requires a singleton over $\bottom/$. This makes sense, because in a sheaf, gluing must be consistent and uniform throughout, and so all gluing has to anchor to a single point.

If we move to presheaves, and hence relax our gluing constraints, then we can have multiple points in the fiber over $\bottom/$. These will correspond to multiple kinds or modes of gluing. This also makes sense, since multiple gluings can only exist in a structure with weaker gluing conditions than we find in a sheaf.

\begin{Example}
\label{ex:multiplicity-over-bottom}

Consider the following simple locale:

\begin{diagram}

% The locale

\node (U) at (0, 3) {$U$};
\node (V) at (-2, 1.5) {$V$};
\node (W) at (2, 1.5) {$W$};
\node (bottom) at (0, 0) {$\bottom/$};

\draw (bottom) to (V);
\draw (bottom) to (W);
\draw (V) to (U);
\draw (W) to (U);

\end{diagram}

\noindent
Now consider a presheaf with more than one anchor over $\bottom/$:

\begin{diagram}

% The locale

\node (U) at (0, 3) {$U$};
\node (V) at (-2, 1.5) {$V$};
\node (W) at (2, 1.5) {$W$};
\node (bottom) at (0, 0) {$\bottom/$};

\draw[dashed] (bottom) to (V);
\draw[dashed] (bottom) to (W);
\draw[dashed] (V) to (U);
\draw[dashed] (W) to (U);

% Fiber over U

\draw (0, 3.25) to (0, 3.775);
\draw (0, 5) ellipse (0.5cm and 1.25cm);
\node[dot, label=above:{$p$}] (Up) at (0, 4) {};
\node[dot, label=above:{$q$}] (Uq) at (0, 4.75) {};
\node[dot, label=above:{$r$}] (Ur) at (0, 5.5) {};

% Fiber over V

\draw (-2, 1.75) to (-2, 2.275);
\draw (-2, 3.1) ellipse (0.375cm and 0.875cm);
\node[dot, label=above:{$a$}] (Va) at (-2, 2.5) {};
\node[dot, label=above:{$b$}] (Vb) at (-2, 3.25) {};

% Fiber over W

\draw (2, 1.75) to (2, 2.275);
\draw (2, 3.1) ellipse (0.375cm and 0.875cm);
\node[dot, label=above:{$c$}] (Wc) at (2, 2.5) {};
\node[dot, label=above:{$d$}] (Wd) at (2, 3.25) {};

% Fiber over bottom

\draw (0, 0.25) to (0, 0.75);
\draw (0, 1.55) ellipse (0.375cm and 0.85cm);
\node[dot, label=above:{$\alpha$}] (alpha) at (0, 1) {};
\node[dot, label=above:{$\beta$}] (beta) at (0, 1.75) {};

% rho U, V

\draw[arrow, ->] (Ur.west) to (Va.east);
\draw[arrow, ->] (Uq.west) to (Vb.east);
\draw[arrow, ->] (Up.west) to (Va.east);
\node at (-1.675, 4.5) {$\restrict{U}{V}$};

% rho U, W

\draw[arrow, ->] (Ur.east) to (Wd.west);
\draw[arrow, ->] (Uq.east) to (Wd.west);
\draw[arrow, ->] (Up.east) to (Wc.west);
\node at (1.675, 4.5) {$\restrict{U}{W}$};

% rho V, bottom

\draw[arrow, ->] (Vb.east) to (beta.west);
\draw[arrow, ->] (Va.east) to (alpha.west);
\node at (-1.15, 1.35) {$\restrict{V}{\bottom/}$};

% rho W, bottom

\draw[arrow, ->] (Wd.west) to (beta.east);
\draw[arrow, ->] (Wc.west) to (alpha.east);
\node at (1.15, 1.35) {$\restrict{W}{\bottom/}$};

\end{diagram}

\noindent
For the cover $\{ V, W \}$ of $U$, take patch candidates $a$ and $c$:

\begin{diagram}

\draw[rounded corners=4pt,fill=selected] (-2.5, 2.65) rectangle (-1.5, 2.35);
\draw[rounded corners=4pt,fill=selected] (1.5, 2.65) rectangle (2.5, 2.35);

% The locale

\node (U) at (0, 3) {$U$};
\node (V) at (-2, 1.5) {$V$};
\node (W) at (2, 1.5) {$W$};
\node (bottom) at (0, 0) {$\bottom/$};

\draw[dashed, faded] (bottom) to (V);
\draw[dashed, faded] (bottom) to (W);
\draw[dashed, faded] (V) to (U);
\draw[dashed, faded] (W) to (U);

% Fiber over U

\draw (0, 3.25) to (0, 3.775);
\draw (0, 5) ellipse (0.5cm and 1.25cm);
\node[dot, label=above:{\textcolor{faded}{$p$}}, faded] (Up) at (0, 4) {};
\node[dot, label=above:{\textcolor{faded}{$q$}}, faded] (Uq) at (0, 4.75) {};
\node[dot, label=above:{\textcolor{faded}{$r$}}, faded] (Ur) at (0, 5.5) {};

% Fiber over V

\draw (-2, 1.75) to (-2, 2.275);
\draw (-2, 3.1) ellipse (0.375cm and 0.875cm);
\node[dot, label=above:{$a$}] (Va) at (-2, 2.5) {};
\node[dot, label=above:{\textcolor{faded}{$b$}}, faded] (Vb) at (-2, 3.25) {};

% Fiber over W

\draw (2, 1.75) to (2, 2.275);
\draw (2, 3.1) ellipse (0.375cm and 0.875cm);
\node[dot, label=above:{$c$}] (Wc) at (2, 2.5) {};
\node[dot, label=above:{\textcolor{faded}{$d$}}, faded] (Wd) at (2, 3.25) {};

% Fiber over bottom

\draw (0, 0.25) to (0, 0.75);
\draw (0, 1.55) ellipse (0.375cm and 0.85cm);
\node[dot, label=above:{\textcolor{faded}{$\alpha$}}, faded] (alpha) at (0, 1) {};
\node[dot, label=above:{\textcolor{faded}{$\beta$}}, faded] (beta) at (0, 1.75) {};

% rho U, V

\draw[arrow, ->, faded] (Ur.west) to (Va.east);
\draw[arrow, ->, faded] (Uq.west) to (Vb.east);
\draw[arrow, ->, faded] (Up.west) to (Va.east);
\node at (-1.675, 4.5) {$\restrict{U}{V}$};

% rho U, W

\draw[arrow, ->, faded] (Ur.east) to (Wd.west);
\draw[arrow, ->, faded] (Uq.east) to (Wd.west);
\draw[arrow, ->, faded] (Up.east) to (Wc.west);
\node at (1.675, 4.5) {$\restrict{U}{W}$};

% rho V, bottom

\draw[arrow, ->, faded] (Vb.east) to (beta.west);
\draw[arrow, ->, faded] (Va.east) to (alpha.west);
\node at (-1.15, 1.35) {$\restrict{V}{\bottom/}$};

% rho W, bottom

\draw[arrow, ->, faded] (Wd.west) to (beta.east);
\draw[arrow, ->, faded] (Wc.west) to (alpha.east);
\node at (1.15, 1.35) {$\restrict{W}{\bottom/}$};

\end{diagram}

These glue at $p$, and are anchored to $\alpha$:

\begin{diagram}

\draw[rounded corners=4pt,fill=selected] (-2.5, 2.65) rectangle (-1.5, 2.35);
\draw[rounded corners=4pt,fill=selected] (1.5, 2.65) rectangle (2.5, 2.35);
\draw[rounded corners=4pt,fill=selected2] (-0.5, 4.15) rectangle (0.5, 3.85);

% The locale

\node (U) at (0, 3) {$U$};
\node (V) at (-2, 1.5) {$V$};
\node (W) at (2, 1.5) {$W$};
\node (bottom) at (0, 0) {$\bottom/$};

\draw[dashed, faded] (bottom) to (V);
\draw[dashed, faded] (bottom) to (W);
\draw[dashed, faded] (V) to (U);
\draw[dashed, faded] (W) to (U);

% Fiber over U

\draw (0, 3.25) to (0, 3.775);
\draw (0, 5) ellipse (0.5cm and 1.25cm);
\node[dot, label=above:{$p$}] (Up) at (0, 4) {};
\node[dot, label=above:{\textcolor{faded}{$q$}}, faded] (Uq) at (0, 4.75) {};
\node[dot, label=above:{\textcolor{faded}{$r$}}, faded] (Ur) at (0, 5.5) {};

% Fiber over V

\draw (-2, 1.75) to (-2, 2.275);
\draw (-2, 3.1) ellipse (0.375cm and 0.875cm);
\node[dot, label=above:{$a$}] (Va) at (-2, 2.5) {};
\node[dot, label=above:{\textcolor{faded}{$b$}}, faded] (Vb) at (-2, 3.25) {};

% Fiber over W

\draw (2, 1.75) to (2, 2.275);
\draw (2, 3.1) ellipse (0.375cm and 0.875cm);
\node[dot, label=above:{$c$}] (Wc) at (2, 2.5) {};
\node[dot, label=above:{\textcolor{faded}{$d$}}, faded] (Wd) at (2, 3.25) {};

% Fiber over bottom

\draw (0, 0.25) to (0, 0.75);
\draw (0, 1.55) ellipse (0.375cm and 0.85cm);
\node[dot, label=above:{$\alpha$}] (alpha) at (0, 1) {};
\node[dot, label=above:{\textcolor{faded}{$\beta$}}, faded] (beta) at (0, 1.75) {};

% rho U, V

\draw[arrow, ->, faded] (Ur.west) to (Va.east);
\draw[arrow, ->, faded] (Uq.west) to (Vb.east);
\draw[arrow, ->] (Up.west) to (Va.east);
\node at (-1.675, 4.5) {$\restrict{U}{V}$};

% rho U, W

\draw[arrow, ->, faded] (Ur.east) to (Wd.west);
\draw[arrow, ->, faded] (Uq.east) to (Wd.west);
\draw[arrow, ->] (Up.east) to (Wc.west);
\node at (1.675, 4.5) {$\restrict{U}{W}$};

% rho V, bottom

\draw[arrow, ->, faded] (Vb.east) to (beta.west);
\draw[arrow, ->] (Va.east) to (alpha.west);
\node at (-1.15, 1.35) {$\restrict{V}{\bottom/}$};

% rho W, bottom

\draw[arrow, ->, faded] (Wd.west) to (beta.east);
\draw[arrow, ->] (Wc.west) to (alpha.east);
\node at (1.15, 1.35) {$\restrict{W}{\bottom/}$};

\end{diagram}

Similarly, the patch candidates $b$ and $d$ glue at $q$, and anchor at $\beta$:

\begin{diagram}

\draw[rounded corners=4pt,fill=selected] (-2.5, 3.4) rectangle (-1.5, 3.1);
\draw[rounded corners=4pt,fill=selected] (1.5, 3.4) rectangle (2.5, 3.1);
\draw[rounded corners=4pt,fill=selected2] (-0.5, 4.9) rectangle (0.5, 4.6);

% The locale

\node (U) at (0, 3) {$U$};
\node (V) at (-2, 1.5) {$V$};
\node (W) at (2, 1.5) {$W$};
\node (bottom) at (0, 0) {$\bottom/$};

\draw[dashed, faded] (bottom) to (V);
\draw[dashed, faded] (bottom) to (W);
\draw[dashed, faded] (V) to (U);
\draw[dashed, faded] (W) to (U);

% Fiber over U

\draw (0, 3.25) to (0, 3.775);
\draw (0, 5) ellipse (0.5cm and 1.25cm);
\node[dot, label=above:{\textcolor{faded}{$p$}}, faded] (Up) at (0, 4) {};
\node[dot, label=above:{$q$}] (Uq) at (0, 4.75) {};
\node[dot, label=above:{\textcolor{faded}{$r$}}, faded] (Ur) at (0, 5.5) {};

% Fiber over V

\draw (-2, 1.75) to (-2, 2.275);
\draw (-2, 3.1) ellipse (0.375cm and 0.875cm);
\node[dot, label=above:{\textcolor{faded}{$a$}}, faded] (Va) at (-2, 2.5) {};
\node[dot, label=above:{$b$}] (Vb) at (-2, 3.25) {};

% Fiber over W

\draw (2, 1.75) to (2, 2.275);
\draw (2, 3.1) ellipse (0.375cm and 0.875cm);
\node[dot, label=above:{\textcolor{faded}{$c$}}, faded] (Wc) at (2, 2.5) {};
\node[dot, label=above:{$d$}] (Wd) at (2, 3.25) {};

% Fiber over bottom

\draw (0, 0.25) to (0, 0.75);
\draw (0, 1.55) ellipse (0.375cm and 0.85cm);
\node[dot, label=above:{\textcolor{faded}{$\alpha$}}, faded] (alpha) at (0, 1) {};
\node[dot, label=above:{$\beta$}] (beta) at (0, 1.75) {};

% rho U, V

\draw[arrow, ->, faded] (Ur.west) to (Va.east);
\draw[arrow, ->] (Uq.west) to (Vb.east);
\draw[arrow, ->, faded] (Up.west) to (Va.east);
\node at (-1.675, 4.5) {$\restrict{U}{V}$};

% rho U, W

\draw[arrow, ->, faded] (Ur.east) to (Wd.west);
\draw[arrow, ->] (Uq.east) to (Wd.west);
\draw[arrow, ->, faded] (Up.east) to (Wc.west);
\node at (1.675, 4.5) {$\restrict{U}{W}$};

% rho V, bottom

\draw[arrow, ->] (Vb.east) to (beta.west);
\draw[arrow, ->, faded] (Va.east) to (alpha.west);
\node at (-1.15, 1.35) {$\restrict{V}{\bottom/}$};

% rho W, bottom

\draw[arrow, ->] (Wd.west) to (beta.east);
\draw[arrow, ->, faded] (Wc.west) to (alpha.east);
\node at (1.15, 1.35) {$\restrict{W}{\bottom/}$};

\end{diagram}

\noindent
The patch candidates $a$ and $d$ cannot glue, because their restrictions to $\bottom/$ do not agree. Here is another place where sheaf theory controls coherence: it prevents gluing in more than one mode at the same time. Glued sections must be glued consistently (in whatever mode they anchor to over $\bottom/$). 

\end{Example}

